% Copyright 1988 by Olsen & Associates (O&A), Zurich, Switzerland.
%
%                   All Rights Reserved
%
%
% Permission to use, copy, modify, and distribute this software and its
% documentation for any purpose and without fee is hereby granted,
% provided that the above copyright notice appear in all copies, and
% that both that copyright notice and this permission notice appear in
% supporting documentation, and that all modifications of this software
% or its documentation not made by O&A or its agents are accompanied
% by a prominent notice stating who made the modifications and the date
% of the modifications.
%
% O&A DISCLAIMS ALL WARRANTIES WITH REGARD TO THIS SOFTWARE AND ITS
% DOCUMENTATION, INCLUDING ALL IMPLIED WARRANTIES OF MERCHANTABILITY AND
% FITNESS.  IN NO EVENT SHALL O&A BE LIABLE FOR ANY SPECIAL, INDIRECT OR
% CONSEQUENTIAL DAMAGES, ANY DAMAGES WHATSOEVER RESULTING FROM LOSS OF
% USE, DATA OR PROFITS, WHETHER IN AN ACTION OF CONTRACT, NEGLIGENCE OR
% OTHER TORTIOUS ACTION, ARISING OUT OF OR IN CONNECTION WITH THE USE OR
% PERFORMANCE OF THIS SOFTWARE OR ITS DOCUMENTATION.
%%%%%%%%%%%%%

\chapter{Procedure Variables}

\section{Purpose}

In the object-oriented programming philosophy upon which the \library is based,
procedure variables play an important role in relation to generic objects.
For details on object-oriented programming, see Chapter~\ref{ObjectsChapter}.

Any procedure can be assigned to a procedure variable; when the procedure is 
later needed, the procedure variable is called in its place.

The library uses many procedure variables.  This chapter explains how the 
use of
procedure variables can aid in the construction of
more independent, more reusable, and more general modules.  It also
connects procedure variables both to the Modula-2 language and
to common real-world applications of procedure variables.

\section{Procedure Variables in Modula-2}

A procedure variable is a variable or a parameter to a procedure. 
The main features of procedure variables are the same in both cases. 

A variable has no defined value until it is assigned a value by some
code inside the initialization body
of an implementation module or inside the implementation code of a procedure
call.  The value of a variable can be reassigned at any time
during runtime.  During the execution of a program, a variable may take on many
different values.

Like all variables, a procedure variable has a type. Its type must be a 
procedure
type.  A procedure type specifies which parameters and return value a procedure
must have in order for Modula-2 to permit its assignment to a procedure
variable or its use as a procedure parameter.

The procedure type declaration is like a record type declaration in that it
combines other declarations which a variety of different modules may have
exported.  It differs in the access protection available to it and in the scope
of its parameters.

\subsection{Procedures as Procedure Variable Values}

An assigned procedure variable always has a procedure as its value.  The
procedure does not have to have been exported, nor does any module other
than the implementation module
in which it was named and coded have to have access to its name.

Code can reassign any procedure which is the value of one procedure variable to
another procedure variable.  Code can also call this procedure --- even from
outside the implementation module in which it was named and coded and even if its
name was never made part of a definition module.

\section{Special Features of Procedure Variables}

The procedure variable combines
\begin{itemize}
 \item
  tighter control over data, by virtue of allowing mixes of read-only,
  read-write, and write-only data;
 \item
  tighter control over scope of access to parameters, because only a procedure
  of the right type will ever get a chance;
 \item
  looser control over information hiding, because assignment to a procedure
  variable or procedure parameter means that a module can call a procedure
  which it never imported; and
 \item
  the ability to take on many different values, from many different modules,
  in the course of execution of the same program.
\end{itemize}

The library frequently uses procedure variables to improve the reusability of
its modules.  It also often associates procedure variables with generic
objects.  The role of procedure variables and generic objects in improving
module reusability is so great that it adds a special dimension to procedure
variables in general.  This role is explained in the next subsections.

\subsection{Procedure Variables and Reusability}

A \newterm{reusable component} is a module or set of modules designed
for ``general'' use; it is designed without prior knowledge of by whom 
it will be \code{IMPORT}ed.

Truly reusable components may supply only a scaffold of support for a given
task; the details are filled in subsequently.  One way to fill in the details 
is
to pass procedures and other parameters to the component.  The component will
export initialization procedures for variables and \newterm{registration
procedures} for procedures. The procedures are then called to fill in the 
details.

The component uses its parameters to start and guide itself.  If it gets lost,
can't make further progress, or should make a decision, the component calls
procedures which its \code{IMPORT}er registered.  The
component might even call procedures registered with other parameters
{\em also\/} registered by its \code{IMPORT}er.

The activity of registering a procedure, which a component calls at some later
moment of execution, is called \newterm{registration}.  The procedure
registered is often called a \newterm{handler} or an \newterm{upcall}.

Procedure variables are especially important for reusable components.
When procedure variables are used, the reusable component will not have to be
modified for subsequent uses.
The required procedure variables are coded in their own module or modules,
separate from the completed, unchanging component modules. 
The new procedures are then registered with the old component at runtime.  
In this way,
the old component works on the new domain defined via
the new procedures.

\subsection{Procedure Variables and Generic Objects}
The procedure which a reusable component calls may need more information
than the reusable component has.  The reusable component
can pass this strange information to the procedure as a \newterm{generic
object}.  A module external to the reusable component passes the generic
object to the reusable component.

The chapter on objects discusses generic objects; essentially, they are
chameleons which can change their type.

The library uses a convention whereby any generic object which is a parameter
in a registration procedure has the name \code{importerObject}. 

The idea behind the \code{importerObject} is that the module which implements
the handler registered with
the reusable component normally will also implement code which supports the
generic object passed to the reusable component.  The assumption usually
holds true because the generic object is a Modula-2 opaque type%
\footnote{See Chapter~\ref{ObjectsChapter}.}.

In order to register the handler and pass the generic object,
the module implementing code for the handler and generic object must
\code{IMPORT} the reusable component.  So the generic object is an ``importer
object''.


\subsubsection{Registration Time Generic Objects}

The generic object may contain additional information needed by the handler
to execute in the right domain.  In this case, the generic object
is passed as a second
parameter to the same registration procedure called to register a handler.

A registration procedure according to this model would look like the following.
\begin{verbatim}
            TYPE
                HandlerProc= PROCEDURE( SysTypes.ANYPTR );

            PROCEDURE Register( 
                    handler        : HandlerProc;
                    importerObject : SYSTEM.ADDRESS 
                );
\end{verbatim}

When \code{handler} is called, it is called with \code{importerObject}
as its generic object parameter.  The call to \code{handler} might 
occur at some interrupt level.

\subsubsection{Calling Time Generic Objects}

The generic object may be passed to the reusable component immediately before 
the component calls an upcall, to communicate information about an event 
which has just
occurred and to prod the reusable component to call the upcall.

In this case, code such as the following might appear:
\begin{verbatim}
            TYPE
                Object;
                UpcallProc= PROCEDURE( SysTypes.ANYPTR );

            PROCEDURE Register( 
                    object : Object;
                    upcall : UpcallProc 
                );

            PROCEDURE Call(    
                    object         : Object;
                    importerObject : SysTypes.ANYPTR 
                );
\end{verbatim}

Upcall \code{upcall} is registered with a particular object, \code{object}.
When an event occurs, \proc{Call} is called with the special event object
\code{importerObject}.  \proc{Call} then calls \code{upcall} with
\code{importerObject} as the \code{SysTpes.ANYPTR} parameter.

\section{Uses of Procedure Variables}

The uses of procedure variables fall into several distinct categories.
The categories are based roughly upon the function of the procedure variable 
and the frequency of its reassignment.  A short discussion of these categories 
follows.

\subsection{Handlers}

\newterm{Handlers} are stored in individual variables, arrays, or linear
lists.  Examples of this type include interrupt handlers and handlers for 
pre-specified categories of
events.

A handler is \newterm{install}ed by calling a procedure dedicated to
installing the handler, and passing the procedure desired to be the handler as
a procedure parameter.  In the library, the term \code{Register} is
frequently used for the installation operation.

Handlers have the following common characteristics:
\begin{itemize}
\item
   A handler is not called by the procedure which installs it.
\item
   A handler is called to handle a specific event or to react to a new
    state of affairs.  Usually this event relates to the program as a
    whole (e.g. termination) or to the system inside which the
    program runs (e.g. interrupts).
\item
   Handlers are normally installed during initialization; the events
    they respond to can occur at any time.  The normal procedure 
    follows:
   \begin{itemize}
   \item
    The module exporting a registration procedure installs a default, or
    \newterm{bootstrap} handler.  This handler is installed only so that
    callers of the module will never find an undefined procedure variable
    as the handler.  If called, it often causes abnormal termination.
   \item
    A module \code{IMPORT}ing the registration procedure installs a handler
    by calling the registration procedure from its initialization body.

    This module also registers a \newterm{termination handler} to de-install
    the handler at termination.
   \item
    The handler installed by the \code{IMPORT}ing procedure remains until
    program termination, at which point the termination handler de-installs
    it.
   \end{itemize}
\item
   The handler may take one or more of the following parameters:
   \begin{itemize}
   \item
         An object passed at the time of its installation.  See the module
         \module{ProgArgs} for examples.

         This object usually holds data defined in the module which implements
         the handler.  It is passed as an additional parameter of the call
         which installs the handler.
   \item
         An object generated when the event or condition is detected.
         
         For instance, if a notice has been established to handle an event,
         then the module which detects the event will call \proc{Notices.Call},
         and pass the generic object \code{importerObject}. \proc{Call}
         will pass this as a parameter to each handler.
   \end{itemize}
\end{itemize}

Handlers are often \newterm{chained}.  In this case, all handlers which
have been installed to handle an event or to react to a new state of affairs
are called.

A module may chain modules explicitly; in this case, the module calls all
handlers.  The calling order may be set by the user to first installed,
first called or to last installed, first called.  See the \module{Notices}
module for an example.

As an alternative, a module may recommend chaining.  In this case, the
module calls the most recently installed handler; this handler should call
the next most recently installed handler, and so on.
See the low-level termination module \module{ProgErrs} for
an example.

\subsection{Actions}

An \newterm{action} is called in response to an event.  Handlers are also
called in response to events. What differentiates an action is that
the event to which it responds is not always recognized by the program and
may not be fixed as a possibility before runtime.  The action chosen to
respond to an event may change frequently, even after program initialization
is complete.

Actions depend on the runtime history of a program; handlers persist more or
less throughout.

Actions are often associated with parsing tables, user interface event
management, and extendible formats.
\begin{itemize}
\item
 In the parsing table case, an action is tied to a token recognized during
 parsing.  When recognized, the action corresponding to the token is called.
\item
 In the user interface event management case, the model is that of a finite
 state machine driven by external events.  When a new external event occurs,
 an action is called; the action may reset which actions will occur
 upon the next external event.
\item
 In the extendible format case, a dispatch table contains format commands and
 corresponding actions.  When a string or file is read and a formatting
 command is called, the table is searched for the command; if the command
 is found, its corresponding action is called.  New (command, action) pairs
 may be added to the table at any time.

 The \module{FIOBase} module uses the (command, action) model to let users
 extend format I/O functionality.
\end{itemize}

As indicated by the table and format cases, the installation of actions
takes on more features than does the installation of handlers, because
the event is more often specified with the action.

Neither a handler nor an action is called by the procedure which installs it.

\subsection{Upcalls}

One application of upcalls occurs when one part of a system implicitly 
communicates with another
level, but can't know about the character of that other level.

The higher level, which knows about the lower level, should \newterm{register}
a procedure with the lower level which remains ignorant of the higher level.
The registration procedure exported by the lower level should take a procedure
as a parameter.  The procedure it receives and stores in a procedure
variable is called an \newterm{upcall}. 

By calling the upcall, the lower level can get information from the higher level, even though it is designed and coded independently.

In network protocols, a typical upcall retrieves the address of a buffer.

Since the upcall returns the information the lower level desires, the lower
level does not have to declare explicitly a common area in which it and a higher
level must share information.  This improves protocol efficiency, as the 
elimination of this common area eliminates the need for the higher level to
copy data for the use of the lower level.

Upcalls also improve the reusability of higher levels.  With upcalls, binding  
an arbitrary higher level to an arbitrary lower level
becomes a matter of a few calls by the higher level to registration procedures.
Without upcalls, this binding can involve serious code modification at the 
higher
level.

Notable characteristics of upcalls are:
\begin{itemize}
 \item
   An upcall helps in the response to an event, but doesn't carry out the
   full response.  It mostly gets needed information.
 \item
   Upcalls are not called by the procedures which install them.
 \item
   Upcalls are usually called by interrupt handlers.
\item
   Upcalls are usually installed during initialization.
 \item
   Upcalls are most useful in network and data transfer protocols, which
   have strict layering requirements.
\end{itemize}

\subsection{Generalized ``Upcall''}

The ``upcall'' analogy can be extended to characterize any use of procedure
variables which involves registration.

Most calls to registration procedures are made by code in modules which
\code{IMPORT} the registration procedure.  The module which exports the
registration procedure can usually be considered a ``lower level'' module
and the module which \code{IMPORT}s the registration procedure a ``higher
level'' module --- in this way, a registered procedure can usually be considered
an ``upcall''.

The term ``upcall'' is more specific than the term ``procedure variable;''
``upcall'' always refers to the \code{IMPORT}ation process and 
to the idea of a lower level calling an upper level.  The idea
is, in general,very powerful.

\subsection{Apply Procedures}

An \newterm{apply procedure} is a procedure parameter which is repeatedly
called by the procedure to which it is passed.  Each time this procedure calls
the apply procedure, it passes different parameters.  The idea is that the apply
procedure will ``apply'' its operation to the ``subparts'' or ``subelements''
of an underlying object.

Apply procedures can be used to interpret sets of strings, as in the case of
\proc{DirIO.List}.  An apply procedure is passed to \proc{DirIO.List}, which  
passes 
a different
directory entry to the apply procedure each time it calls the apply
procedure.

Apply procedures can be used to parse parts of structured files.  In this
model, the file has a general structure.  An \newterm{envelope procedure} is
called and the apply procedure is passed to it.  The envelope procedure opens 
the file.
The envelope procedure and the apply procedure parse the file together.  The
envelope procedure parses as much as it can, covering ``general'' rules.  It
then calls the apply procedure, which continues.  The apply procedure returns,
and the envelope procedure again takes control, finding the end of file or
continuing until it needs to call the apply procedure again.
This technique can be used to efficiently parse files of very rich structures.

\subsection{Methods}

As described in Chapter~\ref{ObjectsChapter}, a \newterm{method} implements an
operation whose arguments and semantics are described by the defined
interface to the object.

Some objects allow a choice of the set of methods which will react to calls
using
the object and its interface.  The set of methods chosen is
contained in the private data records of the object as a group of procedure 
variables.
A call to the object and its interface sets off a call
to one of these procedure variables.  The object's \code{Create} procedure 
sets these procedure variables to values which fit the set of methods
chosen.

\section{Dangers of Using Procedure Variables}

Procedure variables are a powerful addition to the programming environment.
As such, there are dangers associated with their use.


One danger is uninitialized variables.  If an uninitialized procedure variable
is called, the system may go crazy.

Uninitialized procedure pointers in C, when set to null and then called, have
been known to re-execute the C procedure \code{main()}, which starts the main
program.  If this repeats itself a few times, a very interesting stack trace
is generated for the post-mortem debugger to read.
What happened in C may not happen in Modula-2, but the possibility exists.


A second danger is that you may become more confused about what procedures
your program has executed and will execute.

A third danger is the overuse of procedure pointers. 
Procedure pointers should not be used when the case statement is simple 
enough to put in code.  Procedure pointers should also not be used when 
a procedure
would suffice.


