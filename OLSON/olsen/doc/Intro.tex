% Copyright 1988 by Olsen & Associates (O&A), Zurich, Switzerland.
%
%		    All Rights Reserved
%
%
% Permission to use, copy, modify, and distribute this software and its
% documentation for any purpose and without fee is hereby granted,
% provided that the above copyright notice appear in all copies, and
% that both that copyright notice and this permission notice appear in
% supporting documentation, and that all modifications of this software
% or its documentation not made by O&A or its agents are accompanied
% by a prominent notice stating who made the modifications and the date
% of the modifications.
%
% O&A DISCLAIMS ALL WARRANTIES WITH REGARD TO THIS SOFTWARE AND ITS
% DOCUMENTATION, INCLUDING ALL IMPLIED WARRANTIES OF MERCHANTABILITY AND
% FITNESS.  IN NO EVENT SHALL O&A BE LIABLE FOR ANY SPECIAL, INDIRECT OR
% CONSEQUENTIAL DAMAGES, ANY DAMAGES WHATSOEVER RESULTING FROM LOSS OF
% USE, DATA OR PROFITS, WHETHER IN AN ACTION OF CONTRACT, NEGLIGENCE OR
% OTHER TORTIOUS ACTION, ARISING OUT OF OR IN CONNECTION WITH THE USE OR
% PERFORMANCE OF THIS SOFTWARE OR ITS DOCUMENTATION.
%%%%%%%%%%%%%

\chapter{Introduction}

\xquote{Andrew Rooney}{Pieces of My Mind}{
    First you count out all the ads looking for nuclear physicists,
    registered, nurses, animal trainers, and, if you don't know
    anything about computers, you count out the ads looking for 
    computer programmers.   I mention that because there seem to be a
    lot of ads for them these days.  I don't know what they do
    but I assume it's a terrible job that doesn't pay much.
    If it wasn't, there wouldn't be so many ads for them under Help
    Wanted.
}    

Programming is a tough business.  Modula-2 makes a programmer's job
harder by dividing the programming into two distinct tasks.
A Modula-2 programmer
must not only write software but also {\em design} it.
In its primary role, the \library\ supports the task of programming,
not designing.  Since everyone has the right to its source, the
library also helps the Modula-2 designer; it serves as both a good
and bad example of how to design computer programs.\footnote{
    We leave it as an exercise to determine which parts are
    worthy examples of good design practices.
    }
In any form, most designers and programmers should find the source
of the library quite useful.

\section{How to Use This Manual}

Most programmers will be familiar with the general concepts in the
library.  However, we recommend that all programmers skim this manual
at least once.  The chapters are in order of utility.  Some of the
chapters contain useful reference material (such as the
\module{FormatIO} tables in chapter~\ref{IntroIOChapter}).
The appendices may be read independently and need be consulted only 
when the need arises.  The Implementation Notes appendix should be
read before using the library.  

The reader should have a working knowledge of Modula-2.  If you
are new to the language, read the Introduction to Input/Output
chapter and try to implement the examples on your own.  Experience
with other libraries is obviously useful.  If you are quite experienced,
you might be able to get away with just reading the definition modules.

The chapters describe the library generally.
The definition modules cover the specifics.  Many of the modules
are not described in this manual, because their scope does not
extend beyond the borders of a definition module.\footnote{
    This arbitrary boundary was defined so that we could get
    the library released.  Perhaps sometime in the future, these
    other modules will also be covered by the manual.
    }
If you do not find a module in the table of contents, look in the
index.   If the module name is not in the index, it is not mentioned
specifically in any of the chapters.  In this case, the best documentation
will be in the definition module.  
If you get stuck, the \codeterm{examples} and \codeterm{test} directories 
contain many sample uses of the different modules.

\section{\module{SysTypes}}

Most portable software defines a module which translates system
specific type definitions into portable ones.
\module{SysTypes} is that module and is used extensively, hence
it is worthy of a short introduction.
The module \module{SysTypes} defines three classes of declarations:
\term{generic types}, 
\term{storage specific types}, and
\term{type description constants}.

\subsection{Generic Types}
Modula-2 has several pre-defined identifiers which we call
\newterm{generic types}.  Unfortunately, the definitions of some
of these types are too weak to allow the library to gain the most
of various compiler implementations.
The following generic types exported by \module{Sys\-Types}
are used throughout the library in place of their Modula-2 counterparts.
\begin{description}
\item[\codeterm{Card}]\index{\code{Sys\-Types.\-Card}}
    can be used as the index of all possible arrays whose index
    is of the \codeterm{CARDINAL} Modula-2 type.\footnote{
    	By \codeterm{CARDINAL}, we mean any one of:
	\codeterm{CARDINAL}, \codeterm{LONGCARD}, or \codeterm{SHORTCARD}.
    }
    The modules \module{Card\-Conv\-ert}, \module{CardIO},
    and \module{FIOCard} support this type.
\item[\codeterm{Int}]\index{\code{Sys\-Types.\-Int}}
    is of the same order of magnitude of \codeterm{Card}; that is,
    they are in the same storage class.
    The modules \module{Int\-Conv\-ert}, \module{IntIO},
    and \module{FIOInt} support this type.

\item[\codeterm{Real}]\index{\code{Sys\-Types.\-Real}}
    is the largest Modula-2 \codeterm{REAL} type supported by
    the language processor (and likely to be used).
    The modules \module{Real\-Conv\-ert}, \module{RealIO},
    and \module{FIOReal} support this type.
    
\item[\codeterm{SAU}]\index{\code{Sys\-Types.\-SAU}}
    is the smallest addressable unit for the particular Modula-2
    implementation: that is, \codeterm{SYS\-TEM.TSIZE}\code{( SAU )}
    is 1.  The base type of an
    \codeterm{SAU} is a scalar so that it can be used with
    the procedures \proc{ORD} and \proc{VAL}.
    See the implementation module \module{Bytes} for an
    application of this type.

\item[\codeterm{SAUPTR}]\index{\code{Sys\-Types.\-SAUPTR}}
    is defined as a \codeterm{SYS\-TEM.\-AD\-DRESS}.
    Formal parameters of type \codeterm{SAUPTR} are interpreted
    by the procedure as memory (to be copied or otherwise).
    Actual parameters must, therefore, point into valid program memory.
    See the module \module{BinaryIO} for an application of this type.
    
\item[\codeterm{ANY}]\index{\code{Sys\-Types.\-ANY}}
    is the smaller of \codeterm{SYS\-TEM.\-BYTE} and
    \codeterm{SYS\-TEM.\-WORD}.
    This \newterm{magic type} is used in formal parameter
    declarations which accept {\em any type}.  See the module
    \module{Lists} for a sample usage of this generic type.

\item[\codeterm{ANYPTR}]\index{\code{Sys\-Types.\-ANYPTR}}
    is defined as a \codeterm{SYS\-TEM.\-AD\-DRESS}.
    Formal parameters of type \codeterm{ANYPTR} are {\em not}
    interpreted by the procedure in which they are declared.
    Typically, the name of a \codeterm{ANYPTR} is
    \codeterm{impor\-ter\-Obj\-ect} or \codeterm{imp\-Obj}
    to indicate that the object is supplied by the importer
    of this procedure.

\end{description}

\finepoint{
    On most implementations, the \code{FIO} type modules will also
    support all of the other possible types of their class, e.g.
    \module{FIOCard} will support all legal cardinal subranges.
}

\subsection{Storage Specific Types}

The name of a \newterm{storage specific type} identifies its
storage requirements.  Some applications use these types
to conserve space (in program memory or disk files).  A particular
storage specific type may not exist in all implementations; therefore,
programs are less portable when these types are used.   A storage
specific type prefix is the generic type class and its suffix
is its storage class, e.g. \codeterm{REAL32} is of the real
generic type class and occupies 32 bits of storage.  The suffix
of a storage specific type is always in numbers of {\em bits}.

\subsection{Type Description Constants}

A \newterm{type description constant} defines a particular characteristic
of a type.  These constants describe the range and storage specification
of the generic and storage specific types.
The prefix of a
\term{type description constant} identifier defines the characteristic and
the suffix defines the type to which it applies.  The value, of course,
is system dependent.
For example, \codeterm{Sys\-Types.\-MAX\-CARD\-INAL} defines the upper bound of
the type \codeterm{CARD\-INAL}.   The standard prefixes are described
in the \module{Sys\-Types} definition.

\section{Outline}

The other chapters in this manual are:

\begin{description}
\item[Installation]

\item[An Introduction to Input/Output]
    covers the basics of I/O.  \module{SimpleIO} and
    \module{SimpleFIO} are described as well as the general approach
    to \module{TextIO}.
\item[Files]
    describes in detail the semantics of the modules
    \module{BinaryIO} and \module{TextIO}.
\item[Errors and Program Termination]
    defines the semantics
    of errors during I/O, assertion faults, and the
    modules \module{Prog\-Err} and \module{Notices}.
\item[Doing More with FormatIO]
    describes how to implement your own \module{FormatIO} commands.
\item[Objects]
    is a tutorial on object oriented program in Modula-2 and
    describes the module \module{Objects}.
\item[Lightweight Processes]
    introduces the module \module{Tasks}, \module{Interrupts},
    and other task primitives.  

\item[Procedure Variables]
    is a tutorial on Modula-2 procedure types, variables, and
    parameters.

\item[Implementing Modula-2 Classes]
    explains how the library defines {\em classes}
    (\`{a} la Simula, SmallTalk, etc.) in Modula-2.
\end{description}    

