%(****************** P R O P R I E T A R Y   S O F T W A R E ***************)
%(* This software is the property of Olsen & Associates and is proprietary *)
%(* and confidential.  It is not permitted to disclose, copy or distribute *)
%(* the software except by written permission of Olsen & Associates.       *)
%(**************************************************************************)
%
% Copyright 1988 by Olsen & Associates (O&A), Zurich, Switzerland.
%
%		    All Rights Reserved
%
%
% Permission to use, copy, modify, and distribute this software and its
% documentation for any purpose and without fee is hereby granted,
% provided that the above copyright notice appear in all copies, and
% that both that copyright notice and this permission notice appear in
% supporting documentation, and that all modifications of this software
% or its documentation not made by O&A or its agents are accompanied
% by a prominent notice stating who made the modifications and the date
% of the modifications.
%
% O&A DISCLAIMS ALL WARRANTIES WITH REGARD TO THIS SOFTWARE AND ITS
% DOCUMENTATION, INCLUDING ALL IMPLIED WARRANTIES OF MERCHANTABILITY AND
% FITNESS.  IN NO EVENT SHALL O&A BE LIABLE FOR ANY SPECIAL, INDIRECT OR
% CONSEQUENTIAL DAMAGES, ANY DAMAGES WHATSOEVER RESULTING FROM LOSS OF
% USE, DATA OR PROFITS, WHETHER IN AN ACTION OF CONTRACT, NEGLIGENCE OR
% OTHER TORTIOUS ACTION, ARISING OUT OF OR IN CONNECTION WITH THE USE OR
% PERFORMANCE OF THIS SOFTWARE OR ITS DOCUMENTATION.
%%%%%%%%%%%%%

\input{Copyright.tex}
\vspace{10ex}

The Olsen \& Associates Portable Modula-2 Library (henceforth library) was
developed as a part of the Olsen Information System.  The library is
being made available so that other people may benefit from its use.  
We hope that you enjoy it and will forward it to other people who
might find it useful.  Olsen \& Associates expects that the library will
be used in products for profit.  However, our aim is to show that companies
can distribute by-product software free of charge without any
cost to their profits.  In fact, we believe that giving away software
enhances a company's products and reputation and therefore is a natural part
of a sound software marketing strategy.

Olsen \& Associates is not in the business of selling software.  If you
like this software, we would like to find out how much you like it.
The best gauge in a free market system is to either buy our real
service (economic research) or to send us a token of your appreciation.
The reason for this is twofold.  The more people who send us money,
the larger the user community and hence the large the impetus to make
bug fixes.  Monetary contributions will make the people that pay us happy.
If the people that pay us are happy, then we will be able to produce more
freely distributed software.  You may send contributions to:

\begin{quote}
Free Software Fund \\
Olsen \& Associates \\
Research Institute For Applied Economics\\
Seefeldstrasse 233 \\
CH-8008 Z\"urich \\
Switzerland \\
\end{quote}
	
In any event, we hope you will find this software useful.

\medskip
{\bf What's in a name?}

\xquote{Mark Twain}{Life on the Mississippi}{
    A good legible label is usually worth, for information, a
    ton of significant attitude and expression in a historical
    picture.  In Rome, people with fine sympathetic natures stand up
    and weep in front of the celebrated ``Beatrice Cenci the Day
    before Her Execution.''  It shows what a label can do.  If
    they did not know the picture, they would inspect it unmoved,
    and say, ``Young girl with hay fever; young girl with her head 
    in a bag.''
}

The library goes by several names.  The official name is
The Olsen \& Associates Portable Modula-2 Library.
We use the word \newterm{portlib}
internally, because it represents the more portable part of our libraries.
In some places, we use the name \newterm{YAML} (Yet Another Modula-2
Library), because that is exactly what it is.  We do not claim it is
The Standard Library.  It is merely a collection of modules which 
help us program large systems in Modula-2.  The name Olsen \& Associates
Portable Modula-2 Library is a mouthful.  For short, we have another
name --- \newterm{oalib}.  This name exists for the sake of vanity.

\medskip
{\bf History \& Acknowledgements}

This work would not have been possible without the monetary and philosophical
support of Richard Olsen (at the time the library was written he was
sole proprietor of Olsen \& Associates).  He understands the great need
to develop a large body of freely distributable software.  We also thank 
the Free Software Foundation for legitimizing the concept of freely
distributable software.

The I/O part of the library was first designed in September 1985 by
Rob Nagler.  At this time, most other library definitions were neither
portable nor efficient.  The design was based upon the Ad Hoc Working Group's
Library (Modus Issue \#1), the standard Ada library, the Xerox Mesa-Pilot
system, and the standard C library.

Olsen \& Associates started implementing the first Logitech 2.0 prototype
of this basic library in August 1986.  This was done by Jon Bondy and 
Rob Nagler.  The prototype was ported to the Sun by Thomas Beck and
Vince Hatem in December 1986.  The prototype was ported back to the PC
in January 1987 by Jon Bondy.  By May 1987, the library had been ported
back and forth between the PC and Sun enough times that it became stable.
Most of this later work was done by Rob Nagler.

The light-weight process implementation (Tasks et al) was designed and
implemented in April-June 1987.  Tasks was ported to Sun Modula-2 and to
Logitech 3.0 in August-September 1988.  The original design and Sun port
were done by Rob Nagler.  Robert Ward helped port the software to
Logitech 3.0.

Krzysztof Worytkiewicz implemented the current version of the Modula-2
Preprocessor and RealConvert.  Bill Kelly has helped debug many hard
problems.  Bill provided the grounds for and much guidance in the external
design of FormatIO family as well as numerous other modules.  Ion Yadigaroglu
was responsible for designing and implementation ScreenIO and FIOScreen.
Franco Monti and Krzysztof Worytkiewicz were responsible for writing 
many of the test programs.

Any software package would not be complete without quality documentation.
The internal documentation was done by designers and implementers.
Rob Nagler and Toni Wuersch were responsible for the external documentation.
Ion Yadigaroglu was responsible for reformatting the software so that it
would be consistent with our new and improved coding style.  Joanne McKeigue 
battled the LaTeX monster and programmer documentation in order to get the
manual ready for distribution.
\newpage
