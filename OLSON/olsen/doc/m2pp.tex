% Copyright 1988 by Olsen & Associates (O&A), Zurich, Switzerland.
%
%		    All Rights Reserved
%
%
% Permission to use, copy, modify, and distribute this software and its
% documentation for any purpose and without fee is hereby granted,
% provided that the above copyright notice appear in all copies, and
% that both that copyright notice and this permission notice appear in
% supporting documentation, and that all modifications of this software
% or its documentation not made by O&A or its agents are accompanied
% by a prominent notice stating who made the modifications and the date
% of the modifications.
%
% O&A DISCLAIMS ALL WARRANTIES WITH REGARD TO THIS SOFTWARE AND ITS
% DOCUMENTATION, INCLUDING ALL IMPLIED WARRANTIES OF MERCHANTABILITY AND
% FITNESS.  IN NO EVENT SHALL O&A BE LIABLE FOR ANY SPECIAL, INDIRECT OR
% CONSEQUENTIAL DAMAGES, ANY DAMAGES WHATSOEVER RESULTING FROM LOSS OF
% USE, DATA OR PROFITS, WHETHER IN AN ACTION OF CONTRACT, NEGLIGENCE OR
% OTHER TORTIOUS ACTION, ARISING OUT OF OR IN CONNECTION WITH THE USE OR
% PERFORMANCE OF THIS SOFTWARE OR ITS DOCUMENTATION.
%%%%%%%%%%%%%

\chapter{Modula-2 Preprocessor}

\section{Intended Usage}

This preprocessor has a very limited scope.  The purpose is to provide
pre-compile time control of declarations and executable statements in a 
portable manner.  Typically, the preprocessor serves as a version control
mechanism so that the same source file can be used with several
different machines and/or operating systems.   Another use is to 
control the inclusion or exclusion of debugging code in a module.

\section{Input}

There are two types of input to the preprocessor: \newterm{source code}
and \newterm{control variables}.  Source code is normal Modula-2
code inseminated with preprocessor statements.  Control
variables are Modula-2 identifiers which indicate boolean values.
If a control variable is passed to the preprocessor, its
value is \newterm{true}.  If a control variable does not exist,
the preprocessor treats its value as \newterm{false}.  In other
words, a control variable's existence indicates its value.

\section{Output}

The preprocessor parses the source code for preprocessor statements.
If preprocessor statements are not found, the output is {\em identical}
to the input.  There is only one type of preprocessor statement:
the \newterm{If-Then-Else-End}.  The syntax is identical to Modula-2 
except for the \code{@} characters as shown in the following example.

\begin{verbatim}
    Other Modula-2 Code;
    @IF SomeIdentifier THEN
        Some Modula-2 Code;
    @END
    More Modula-2 Code;
\end{verbatim}

The semantics of this statement are quite simple.  If \code{SomeIdentifier}
is \newterm{true}, then \code{Some Modula-2 Code;} is included in 
the output file as follows:

\begin{verbatim}
    Other Modula-2 Code;
    
        Some Modula-2 Code;
    
    More Modula-2 Code;
\end{verbatim}

If \code{SomeIdentifier} is \newterm{false}, the output looks like:

\begin{verbatim}
    Other Modula-2 Code;
    

    
    More Modula-2 Code;
\end{verbatim}

There are two important things to note.  First, preprocessor statements 
are always eliminated during preprocessing.  Second, the number of lines
in the output is always equal to the number of lines read from the
source code.  Therefore, source code oriented diagnostic output (from
compilers or debuggers) can always be directed at the source code
as opposed to having to use the preprocessed versions of the source
code..

The \newterm{Else-Clause} and \newterm{Elsif-Clause} are 
optional (as demonstrated in the previous
example).  They behave just as in Modula-2.  If the expression between
the \code{@IF} and the \code{THEN} is true, then the code in between
the \code{THEN} and the \code{@ELSE} is included and the code between
the \code{@ELSE} and the \code{@END} is not included.  If the expression
is false, the code after the \code{THEN} is not included and the
code after \code{@ELSE} is included.

\newterm{Expressions} are combinations of identifiers and operators.  
The operators are the normal Modula-2 boolean operators:
\code{AND}, \code{OR}, and \code{NOT}.  The expressions may be
parenthesized to force a specific interpretation order.  The
default precedence is identical to that of Modula-2.  

\newterm{Identifiers} indicate truth values based on their existence
(as made known to the preprocessor).  The preprocessor is initialized
with a set of \newterm{true} identifiers; the value true
is used in place of the identifier in expressions.
If an expression contains an identifier which is not in this list,
its value is taken to be {\em false}.
Two keywords have been defined for the programmer's convenience:
\code{TRUE}\index{\code{TRUE}} and \code{FALSE}\index{\code{FALSE}}.
These variables always represent the logical values true and false, 
respectively.

Preprocessor statements may be nested as in Modula-2.  They may
appear anywhere and may be indented or distributed across several
lines.  The following are all legal examples of preprocessor statements,
but they are highly unorthodox and not a recommended style convention.
\begin{verbatim}
    @IF SomeExpression THEN Some Modula-2 Code @END

    @IF SomeLongExpression
      THEN
    @END

    @IF
        SomeWeirdExpression
            THEN
                @ELSE
                    @END
\end{verbatim}


\section{Usage}

There are two ways to use the preprocessor: as a command
and as a Modula-2 module.  The command is called \code{m2pp}\index{m2pp}.
There are several ways one can execute \code{m2pp}, as follows.

\newcommand{\PromptedCommand}[1]{{\tt yfp\% \underline{#1}}}
\begin{quote}
    \PromptedCommand{m2pp input output}
\end{quote}
or
\begin{quote}
    \PromptedCommand{m2pp input > output}
\end{quote}
or
\begin{quote}
    \PromptedCommand{m2pp < input > output}
\end{quote}

\subsection{Options}
The following options are defined:
\begin{description}
    \item{\tt D {\em variable}}
	sets the specified {\em variable} to {\em true}.
    \item{\tt U {\em variable}}
	sets the specified {\em variable} to {\em false}.
    \item{\tt w}
	makes the output file writable.  This option is defined only
	for the Sun implementation.  The output file must be specified
	on the command line (as opposed to being redirected).  By default
	on the Sun, output is read-only to avoid mistakes editing the
	wrong files.
    \item{\tt Strip}
	eliminates comments from the source.  This feature can be useful
	for disk space conservation.   If this option is used, comments
	with semantic value that being with a `\code{\$}' immediately
	following the comment bracket are {\em not stripped}.
\end{description}

On the Sun, arguments are passed in the form:
\begin{quote}
    \PromptedCommand{m2pp -DTrue1 -D True2 -UFalse1 input output}
\end{quote}
The {\em space} between the \code{D} and \code{True2} is not required,
because the module \module{ProgArgs} has a flexible format for single
letter flagged values.

On MS-DOS, arguments are passed as follows:
\begin{quote}
    \PromptedCommand{m2pp/D=True1/D=True2/U=False1 input output}
\end{quote}
Given the command line limitations of the MS-DOS, you probably want
to use the \code{M2ppVars} variable described in the next section.

\subsection{Environment}
\index{\code{M2ppVars}, environment variable}
The environment variable \code{M2ppVars} can be defined by the user.
\code{M2ppVars} specifies a list of {\em true} control variables to
be used by \code{m2pp}.  On the Sun, this variable is of the form:
\begin{verbatim}
    M2ppVars=True1:True2:True3
\end{verbatim}
On the PC,
\begin{verbatim}
    M2PPVARS=True1;True2;True3
\end{verbatim}

\code{m2pp} reads \code{M2ppVars} {\em before} reading the command line
arguments \code{U} and \code{D}.

\section{Preprocessing from a Modula-2 Program}

The module \module{M2PParser} and \module{M2PScanner} provide 
programmatic access to
the preprocessor.  The following facilities are provided:
\begin{itemize}
\item
    Control variable truth value control on a per file basis
\item
    Integrated multiple file preprocessing (including standard input)
\end{itemize}
The definition modules provide an exact explanation of the entry points
and facilities provided.

\section{Syntax}

In the interest of thoroughness, the Extended Backus Naur Form\index{Extended
Backus Naur Form}(EBNF)
of the preprocessor language follows.

\begin{description}

\item[\newterm{Statement}]
    = \newterm{If-Then-Else-End} {\tt |} 
      \{ \newterm{Other} \} \{ \newterm{Statement} \} \{ \newterm{Other} \} 

\item[\newterm{If-Then-Else-End}]
    = \code{@IF}\newterm{Expression} \code{THEN}  
      \{ \newterm{Other} \} \{ \newterm{Statement} \} \{ \newterm{Other} \} 
      \{ \newterm{Elsif-Clause} \}
      \{ \newterm{Else-Clause} \} \code{@END}.

\item[\newterm{Elsif-Clause}]
    = \code{@ELSIF} \newterm{Expression} \code{THEN}
      \{ \newterm{Other} \} \{ \newterm{Statement} \} \{ \newterm{Other} \}.

\item[\newterm{Else-Clause}]
    = \code{@ELSE} \{ \newterm{Other} \} 
      \{ \newterm{Statement} \} \{ \newterm{Other} \}.

\item[\newterm{Expression}]
    = \newterm{OrFactor}.

\item[\newterm{OrFactor}]
    = \newterm{AndFactor} \{ \code{OR} \newterm{AndFactor} \}.

\item[\newterm{AndFactor}]
    = \newterm{NotFactor} \{ \code{AND} \newterm{NotFactor} \}.

\item[\newterm{NotFactor}]
    = \newterm{Identifier} {\tt |} \code{TRUE} {\tt |} \code{FALSE} {\tt |} 
      \code{NOT} \newterm{NotFactor} {\tt |}
      \code{(} \newterm{OrFactor} \code{)}.

\item[\newterm{Identifier}]
    = Modula-2 identifier.

\item[\newterm{Other}]
    = Any sequence of characters except ASCII null.

\item[\newterm{ReservedTokens}]
    = \code{@} {\tt |} \code{IF} {\tt |} \code{THEN}{\tt |} \code{ELSIF} 
      {\tt |} \code{ELSE}  
      {\tt |} \code{END} {\tt |} \code{(} {\tt |} \code{)} {\tt |}
      \code{FALSE} {\tt |} \code{TRUE} {\tt |} \code{AND} {\tt |} 
      \code{OR} {\tt |} \code{NOT}.

\end{description}

\section{File Names}
    \index{Preprocessor File Names}
    \index{\code{m2pp} File Names}
Since Modula-2 preprocessor input is not Modula-2, a different file naming
convention\index{file naming convention}\index{convention, file naming} has 
been chosen.  The file name suffix replacements
to be used are:
\begin{description}

\item[\codeterm{.mpp}]
    in place of \codeterm{.mod}

\item[\codeterm{.dpp}]
    in place of \codeterm{.def}

\end{description}

\section{Preprocessor Controls}\index{preprocessor control}
There are several classes of preprocessor control:

\begin{itemize}
	\item
	Portability -- e.g. LittleEndian
	\item
	Feature Inclusion -- e.g. Tasks
	\item
	Bug Exclusion -- e.g. SunBug
	\item
	Code Efficiency -- e.g. Assert
	\item
	Test Code -- e.g. Simulate
\end{itemize}

\section{Variable Names}
    \index{Preprocessor Variable Names}
    \index{\code{m2pp} Variable Names}
    
The following list defines all of the \term{Modula-2 Preprocessor}
variable names used by
the library and then some.  The policy used throughout the library
is to use the weakest form of the name possible.\footnote{
    The current implementation of the library is errant in this
    respect, because the major use of the \codeterm{SunOS} 
    variable is for \codeterm{UnixOS} features, i.e. \codeterm{UnixOS}
    is weaker than \codeterm{SunOS}.
    }
For example, if a feature is applicable to all version of the
IBM PC architecture, the variable \codeterm{IbmPcArch} should
be used as opposed to \codeterm{IbmPcAtArch} even though the 
software will probably be only run on an IBM PC AT.  

\begin{description}
\newcommand{\codedesc}[1]{\item[\codeterm{#1}]}

\codedesc{Assert} -- Assertion checking code

\codedesc{BigEndian} -- Low address contains most significant byte

\codedesc{Debug} -- Debugging code

\codedesc{FALSE} -- Always undefined

\codedesc{IbmPcArch} -- all IBM PC architectures

\codedesc{LittleEndian} -- Low address contains least significant byte

\codedesc{LogitechM2} -- all Logitech Modula-2 language processors

\codedesc{LogitechM2V2} -- Logitech 2.X Modula-2 language processor

\codedesc{LogitechM2V3} -- Logitech 2.X Modula-2 language processor

\codedesc{MC68881Arch} -- Motorola 68881 floating point chip

\codedesc{MsDosOS} -- All MS-DOS (PC-DOS) operating systems

\codedesc{M2OnePass} -- Requires \codeterm{FORWARD} statements

\codedesc{M2V2} -- Version 2 Modula-2 as defined by Wirth

\codedesc{M2V3} -- Version 3 Modula-2 as defined by Wirth

\codedesc{NoNotices} -- The module \module{Notices} is not used

\codedesc{NoStorage} -- The module \module{Storage} is not used

\codedesc{SunArch} -- all Sun architectures

\codedesc{Sun3Arch} -- Sun 3 hardware

\codedesc{Sun4Arch} -- Sun 4 hardware

\codedesc{SunM2} -- Sun Modula-2 language processor

\codedesc{SunOS} -- all Sun operating systems

\codedesc{Sun3OS} -- Sun 3.X operating systems

\codedesc{Sun4OS} -- Sun 4.X operating systems

\codedesc{Tasks} -- Ligthweight process code

\codedesc{TRUE} -- Always defined

\codedesc{U3bOS} -- AT\&T 3b architecture

\codedesc{U3b5OS} -- AT\&T 3b5 architecture

\codedesc{UnixOS} -- all Unix operating systems

\codedesc{UnixBsd4d2OS} -- Berkeley 4.2 Unix operating system 

\codedesc{UnixAttVOS}  -- AT\&T System V Unix operating system

\codedesc{VaxArch}     -- Vax architecture

\end{description}

\section{Known Bugs}
There is a problem with extremely long lines and comment strip mode.
In the event the `\code{(}' and `\code{*}' are in two separate buffers,
the comment will not be stripped.  Comments with semantic meaning will
be stripped if the `\code{\$}' is in a separate buffer from the comment
bracket `\code{(*}'.  The buffer size is 
\codeterm{GenConsts.maxLineLength} (which in the distributed version of
the library is 100 characters).
