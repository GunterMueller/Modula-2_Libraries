% Copyright 1988 by Olsen & Associates (O&A), Zurich, Switzerland.
%
%		    All Rights Reserved
%
%
% Permission to use, copy, modify, and distribute this software and its
% documentation for any purpose and without fee is hereby granted,
% provided that the above copyright notice appear in all copies, and
% that both that copyright notice and this permission notice appear in
% supporting documentation, and that all modifications of this software
% or its documentation not made by O&A or its agents are accompanied
% by a prominent notice stating who made the modifications and the date
% of the modifications.
%
% O&A DISCLAIMS ALL WARRANTIES WITH REGARD TO THIS SOFTWARE AND ITS
% DOCUMENTATION, INCLUDING ALL IMPLIED WARRANTIES OF MERCHANTABILITY AND
% FITNESS.  IN NO EVENT SHALL O&A BE LIABLE FOR ANY SPECIAL, INDIRECT OR
% CONSEQUENTIAL DAMAGES, ANY DAMAGES WHATSOEVER RESULTING FROM LOSS OF
% USE, DATA OR PROFITS, WHETHER IN AN ACTION OF CONTRACT, NEGLIGENCE OR
% OTHER TORTIOUS ACTION, ARISING OUT OF OR IN CONNECTION WITH THE USE OR
% PERFORMANCE OF THIS SOFTWARE OR ITS DOCUMENTATION.
%%%%%%%%%%%%%

\chapter{Coding Style}

\section{Introduction}

Coding standards are prevalent throughout the industry to support
maintainability and to ease development.  This document describes a coding
standard which is based upon the personal opinions of the O\&A software group 
and conforms to other industry standards.
Most of the guidelines are oriented towards code readability.

\section{Bill of Rights}

\begin{enumerate}
\item
    Any of the following guidelines can be broken if they pass through a code
    review without objection (not by oversight, however).
\item
    The placement of the vertical bar `\code{|}' separator in 
    \code{CASE} statements are not governed by these guidelines.
\end{enumerate}

\section{Guidelines}
\begin{enumerate}
\item
    Naming
    \begin{enumerate}
    \item
        Variable names should pass the following criterion:

        ``If a reviewer has to ask what an identifier means, 
        it fails the comprehendability test.  The name should be changed 
        or its declaration should contain a clearer description.''
    \item
        Identifiers are of mixed case where upper case letters
        separate words.  To make code more readable, the case of the first 
        letter of the identifier indicates its class.  
        Modules, procedures, and types should begin with an upper case letter.
        Variables, constants (enumeration values), and fields should begin
        with a lower case letter.  Names of program modules are excluded 
        from this naming convention, because they fall into another category
        of naming conventions.
    \item
        When developing a family of modules, a particular 
        sequence of characters which is unique within the family scope should
        be chosen to identify modules from this family.  In naming globally
        available identifiers (be they files or FormatIO types), the family
        identifier should be part of these new identifiers.  For example,
	the time modules all have the keyword \code{Time} in their names.

    \item
        Names should not contain redundant parts unless this
        redundancy supports a clear meaning.  Exported identifiers should
        be treated by this guideline when combined with the module name; i.e.
        \code{Module.Export} does not contain duplicated parts, but the
	name \code{Module.ModuleExport} does.
    \item
	The use of the keyword \code{PRIVATE} indicates limited
	private objects, i.e. entities which cannot be defined as
	a Modula-2 opaque type.
	This naming convention specifies that the
	entity exported is to be used by a restricted set of importers
	or in a limited way.  For example, the \code{BinaryIO.Index}
	field is declared as \code{PRIVATE} because the insides of an
	index variable are implementation dependent but importers
	which need to write them in files need to know its size.
    \end{enumerate}
\item
    Comments:
    \begin{enumerate}
    \item
        Comments are highly recommended.
    \item
        Header comments for implementation modules and  exported 
	procedures should describe 
        the {\em implementation} of these entities and not their function.   
	The definition
        module describes the function.  Copying the declarations from the 
	definition modules
        to the implementation modules as comments is not recommended, 
	because such comments
        are difficult to maintain and sometimes confusing.
    \item
        Clearly named identifiers, e.g. with few abbreviations, can 
	be used in place of
        commenting in certain cases.  The programmer should use ``good 
	judgement''.
    \item
        Comments and code should agree.  Incorrect comments should be 
	removed, because they are negative in value.
    \end{enumerate}
\item
    Declarations:
    \begin{enumerate}
    \item
        A module should be structured in the form:
        \begin{verbatim}
(* Proprietary or Public domain notice except for test programs. *)
    MODULE Name;
    (*
     * Module header comment.
     *)
    CONST
        value = 1;
    TYPE
        SomeType = POINTER TO CARDINAL;
    VAR
        global : SomeType;

    PROCEDURE Foo;
        BEGIN (* Foo *)
        END Foo;

    BEGIN (* Name *)
        SomeProc;
    END Name.
        \end{verbatim}
        In order to support one pass compilers, the standard method of
	module layout should be declaration before first 
	use.  
        It is {\em ok}
        to declare entities more closely to their first use; we do
        not need to stick with Pascal style declaration conventions.

    \item
    Record declarations should look like the following:
        \begin{verbatim}
    TYPE
        SomeRec = RECORD
            field1    : CARDINAL;  (* keeps track of fieldNum2 *)
            fieldNum2 : CARDINAL;  (* Solves the problems of field1 *)
        END;
        \end{verbatim}
        as opposed to 
        \begin{verbatim}
    TYPE
        SomeRec = RECORD
                     field1    : CARDINAL;
                     fieldNum2 : CARDINAL;
        END;
        \end{verbatim}
    \item
    Procedure {\em type} declarations should look like:
        \begin{verbatim}
    TYPE
        SomeProc = PROCEDURE(
            CARDINAL,       (* number of times to jump rope *)
            SYSTEM.ADDRESS  (* where to store the result *)
        ) : BOOLEAN         (* FALSE => Terminate upcall sequence. *)
            (*
             * Comment describing required semantics.
             *)
        \end{verbatim}

    \item
        Procedure declarations should look like record declarations:
        \begin{verbatim}
    PROCEDURE Foo(
            param1 : SomeType;      (* Must be valid *)
            param2 : SomeOtherType; (* Will be appended to param1 *)
        VAR param3 : SomeOtherType  (* The ultimate answer *)
        )          : ReturnType;    (* Degree of ultimacy of param3 *)
        (*
         * Header comment
         *)
        VAR
            local1 : TheType;   (* Temporary counter *)
            local2 : TheType;   (* Number of times to compute param3 *)
        BEGIN (* Foo *)
            SomeOtherProc;
        END Foo;
        \end{verbatim}
        Note that the entire declaration is indented from the keyword 
        \code{PROCEDURE}.  This allows easy searches for variables
        declared at a particular scope level.   The parameter descriptions
	may appear to the right of their declarations or in the header comment.
	The header comment should appear {\em after} the procedure declaration,
	not before.
    \end{enumerate}
\item
    Formal parameters:
    \begin{enumerate}
    \item 
        Related parameters should be grouped with the primary parameter
        appearing first; modifiers (or qualifiers) appear
	{\em after} the things
        they modify.  Look at the module \code{Strings} for an example.
    \item
        If the procedure's function is like a ``well known'' infix operator,
        the parameters should be positioned in the same order as if the
        operator were used.  For examples, see the module \code{Card32}.
    \item
        Pure {\em out parameters} should appear at the end of the parameter
        list.  If an out parameter determines the validity of the operation,
        it should be the procedure \code{RETURN} value.  For the
	convenience of the importer, return values should be used when
	the procedure has a single output which will be
	used in an expression or as a parameter.  Some examples are
	\code{TextIO.GetOutput}, \code{DirIO.CreateEntry}, and
	\code{Strings.Length}.
    \item
        If efficiency of structured parameters is required, a
        declaration of the following form should appear:
        \begin{verbatim}
    PROCEDURE Something(
        (*IN*)VAR param : SomeLargeRecord
        );
        \end{verbatim}
        In this case, the caller expects the parameter to be treated as
        if it were {\em passed by value} when in reality it has been 
        {\em passed by reference}.  An alternative is to declare
        a pointer type for \code{SomeLargeRecord} which puts more of the
        {\em hack} code in the client.  {\em Note:} do not use this unless
        you are {\em sure} that performance of parameter passing is an
        issue.
    \item
        \code{ARRAY OF CHAR} should be the type of a formal parameter
        when the actual parameter may be one of several CHAR types {\em or}
        it can be expected that the actual parameter may be a constant.
        Usually, \code{ARRAY OF CHAR} is the appropriate choice for input
        parameters.  Output parameters are normally declared with a simple 
        type.   However, if the output parameter is the result of
        an operation which is of an undetermined size or
        which is based upon an input of an unspecified length,
	it should be defined as an open array.
    
    \item
        Procedure types which take an {\em importer object} should
        be declared so that the importer object is the first parameter
        passed to the procedure variable.  This convention agrees with
        the {\em object-oriented} module section.

    \item
        An enumeration type should be used instead of the
        type \code{BOOLEAN} for bi-state formal parameters in exported
	procedure declarations.  The purpose of this is to make the actual
	parameters to said procedures more readable without the need
	for caller-supplied supporting documentation.

    \end{enumerate}
\item
    \code{DEFINITION} modules 
    \begin{enumerate}
    \item
        The description of the module should be detailed enough so that the
        importer does not have to read the code to figure out restrictions
        and caveats.
    \item
        All declarations should be commented (except the primary object,
        assuming that its definition is made apparent in the module header).
    \item
        The semantics of a procedure should be clearly defined including
        careful descriptions of its parameters, either in a general comment
        or by individual descriptions.
    \item
        Exported variables should have a limited scope of access.
        Each time a variable is exported, the reason should be carefully 
        examined (by two reviewers if possible).  This guideline supports the 
	tracing of variable modifications and better control of globally 
	available values in the light-weight
	process environment.  Normally,
	exported global values should be controlled by \code{SetXXX} 
	and \code{GetXXX} procedures, because this allows the implementation of
	mutual exclusion when necessary, without changing the definition
	module.
    \item
        Preprocessor control in definition modules should be limited to
        architecture and operating system flag.  External features
        which change as the result of preprocessor control (e.g. \code{Tasks})
        should appear only in the implementation modules or by in/exclusion
        of whole modules. This allows features (such as tasks) to be included
        or excluded by merely re-linking.
    \item
        Export lists should be tabular without explanations of the 
        types of objects and what they do.  Export lists have been removed from
        the definition of Modula-2 (they are optional).  Long descriptive export
        lists serve only to obscure the definitions of the actual objects being
        exported.  The lists should be in declaration or alphabetic order
	for ease of maintenance.  Exceptionally long export
        lists may include comments, but the usage should pass reviews.
        {\em Note}: until version 3 compilers are more readily available, all
        {\em potentially} portable modules should contain an 
	\code{EXPORT} statement.  An alternative approach is to sort
	the export list alphabetically.

    \end{enumerate}

\item
    \code{IMPLEMENTATION} and program modules:
    \begin{enumerate}
    \item
        The use of local modules is discouraged.  A module is a module 
	and, since we
        don't have generic modules (\`{a} la Ada), there is no point in a module
        having children.  

    \item
        The \code{END} of a control statement should align with its initial line.

    \item 
        \code{BEGIN} should contain a comment stating to which module or 
	procedure
        it belongs.  Note that empty initialization code bodies do not need
        a \code{BEGIN} in Modula-2.
    \item
        Indent by 4; tab stops should be every 8 spaces.

    \item
        Semicolons (;) should be treated as statement terminators not 
	separators even though Modula-2 allows the omission of semicolons
	in certain places.  This makes the code easier to rearrange.

    \item 
        Each statement (or part thereof in the case of
        compound statements) should appear on a separate line unless it
	is much clearer to combine them. 
        Multiple statement lines are okay in 
	initialization sections.
        Declarations should follow this same guideline so that
	each newly defined
        identifier may contain a side description.

    \item
        Initialization statements should be presented in the order of 
	declaration, e.g.
	a record variable's fields.

    \item 
        If a module does type casting (not conversions), it should import
        \code{SYSTEM}, even if it does not use anything 
	exported by this module.  
        This will clearly identify modules which are not portable.  
	Type casting used in case variant records is not excepted!
	{\em Note:} type casting should be avoided wherever possible.

    \item
        If a module outputs {\em debugging} information, it should 
	be controlled by the preprocessor flag \code{Debug} as well 
	as by a boolean or cardinal selectable value (see the module 
	\code{DebugFlags}).  In the header of the module, define the
	following:
        \begin{verbatim}
    @IF Debug THEN
        VAR
     	    debug : BOOLEAN;
    @ELSE
        CONST
	    debug = FALSE;
    @END (* Debug *)
        \end{verbatim}
	 To output debugging information, do the following:
        \begin{verbatim}
    IF debug THEN
        SomeDebugOutput;
    END;
        \end{verbatim}
	Note that SomeDebugOutput must be available even if the
	\code{Debug} flag is off.
    

    \item
        \code{HALT}s should not appear directly in the
        code except in special cases.  The use of 
	% Needed, because LaTeX chops off the last of ProgErr.Terminate.
	\newline
	\code{ProgErr.Terminate}
	is recommended for the majority of program aborts.
    \end{enumerate}

\item
    \code{IMPORT} statements:
    \begin{enumerate}
    \item
        The use of qualified imports in the design of definition modules
	and their use by importers is encouraged.
    \item
        Declarations should look like the following:
        \begin{verbatim}
    IMPORT
        Module1,     
        Module2,     
        Module3,     
        Module4,  
        Module5;
        \end{verbatim}
        The Modules should be in alphabetical order for easy maintenance.

    \end{enumerate}
    
\item
    An ``object-oriented'' module operates on a particular type which
    may or may not be exported by the module itself.  The following guidelines
    apply to this class of modules:
    \begin{enumerate}
    \item
        The primary opaque type, if exported by the module, should be named
        \code{Object}.  When combined with the name of the module, this
        usually defines clearly what the type should be. Some
        examples are: \code{TextIO.Object}, \code{BinaryIO.Object},
        \code{Lists.Object}, etc.  If there is no primary opaque type
	or the name \code{Object} is confusing, a more descriptive
        name should be chosen, e.g. \code{Time.Point}, \code{Time.Delta},
        \code{Objects.Manager}, and \code{BinaryIO.Index}.
    \item
        \code{Create} {\em or} \code{Open}  
         initializes or creates the object.
    \item
        \code{Destroy} {\em or} \code{Close} deallocates the object.
    \item
        \code{IsValid} {\em or} \code{GetState} 
        determines (non-destructively) whether or not the object has been 
        created.
    \item
        \code{SetInvalid} initializes the object to a guaranteed
        invalid state.
    \item
        The object should be implemented using \code{Objects}.
    \item
        All exported interfaces except the initialization procedures
        should call \code{Objects.\-Assert} or \code{Objects.IsValid}.
    \item
        Given that Modula-2 variables can be uninitialized when used,
        the module should allow the destruction of an object which is already
        invalid.  No checks on the validity of the object
        should be made by the initialization procedures.
    \item
        Importers of real opaque types should make no assumptions about the
        contents of the real opaque types except that their size is the same 
	as a \code{SYSTEM.ADDRESS}.
        The size of limited private types should not be assumed to be
        the same as a \code{SYSTEM.ADDRESS}.  Using the contents of 
	an opaque type
        makes code immediately non-portable and the documentation
        should clearly state the reason for doing so.
    \item
        The main object of the module should appear as the first
        parameter to all the procedures in which the object type 
        appears as the parameter.  If there are several objects, 
        then pick one to be the main one.
    \end{enumerate}

\item
    Some philosophy:
    \begin{enumerate}
    \item
        Modules should be object oriented where possible.  This
        is usually not the most obvious route, but it turns out to be the 
	easiest
        in the long term.  
    \item
        If a module calls anonymous procedures (i.e. upcalls),
        it should save the state of the global variables before making the
        upcalls.  Normally, it should be the last thing done
        in the procedure before returning.  This avoids problems with
        assumptions made by the importer.
    \item
        Global state (other than contained in objects) should be avoided.
        Using variables which hold state across procedure calls is 
        dangerous in Modula-2, because modules are often used for more
        things than those for which they were originally intended.  
    \item
        Look around to see if has been done before.  Software is very
        often highly reusable.  If someone has solved a similar problem,
        chances are you can use part of it.   If nothing else, evaluate
        the ideas contained within the other software so that the new
        software doesn't make the same mistakes as the old.
    \item
        Types and constants should be defined where they
        are first used or are primarily implemented, e.g. \code{Time} 
	implements \code{Point} and \code{Delta}, thus they are
	declared in \code{Time}.
	If you {\em need} to avoid circular imports, use an \code{XXXConsts} 
        module
        which defines the constants 
        and types for a particular group of modules when defining types.  
    \item 
        During abnormal termination, only critical resources should be
        reset (e.g. serial ports).  This will ensure the ability of the
        application to bring itself down gracefully.
    \end{enumerate}

\end{enumerate}

