% Copyright 1988 by Olsen & Associates (O&A), Zurich, Switzerland.
%
%		    All Rights Reserved
%
%
% Permission to use, copy, modify, and distribute this software and its
% documentation for any purpose and without fee is hereby granted,
% provided that the above copyright notice appear in all copies, and
% that both that copyright notice and this permission notice appear in
% supporting documentation, and that all modifications of this software
% or its documentation not made by O&A or its agents are accompanied
% by a prominent notice stating who made the modifications and the date
% of the modifications.
%
% O&A DISCLAIMS ALL WARRANTIES WITH REGARD TO THIS SOFTWARE AND ITS
% DOCUMENTATION, INCLUDING ALL IMPLIED WARRANTIES OF MERCHANTABILITY AND
% FITNESS.  IN NO EVENT SHALL O&A BE LIABLE FOR ANY SPECIAL, INDIRECT OR
% CONSEQUENTIAL DAMAGES, ANY DAMAGES WHATSOEVER RESULTING FROM LOSS OF
% USE, DATA OR PROFITS, WHETHER IN AN ACTION OF CONTRACT, NEGLIGENCE OR
% OTHER TORTIOUS ACTION, ARISING OUT OF OR IN CONNECTION WITH THE USE OR
% PERFORMANCE OF THIS SOFTWARE OR ITS DOCUMENTATION.
%%%%%%%%%%%%%

\chapter{Implementation Notes}

\xquote{Plato}{Apology}{
    Nor do I converse only with those who pay;
    but anyone, whether he be rich or poor, may ask and answer
    me and listen to my words; and whether he turns out to be a
    bad man or a good one, neither result can be justly imputed to
    me; for I never taught or professed to teach him anything.
    And if anyone says that he has ever learned or heard anything
    from me in private which all the world has not heard,
    let me tell you that he is lying.
}    

This post haste collection of thoughts is designed to aid the
gentle programmer in the task of using the \library.  Naturally,
any such collection is incomplete.  At times you might find 
difficulty in understanding why a particular feature\footnote{
    Known in some circles as a bug} 
is implemented the way it is.  There is an order to figuring
out the rhymes and reasons:
\begin{enumerate}
\item
    Look in the other chapters in this manual for a discussion
    of the feature.
\item
    Look in this chapter for some nitty gritty discussion.
\item
    Look in the definition module(s) which export the feature.
    If this fails,
\item
    Look in the implementation module(s) which implements the feature.
\item
    If all of the above fail to explain the unexpected behavior, 
    the feature may actually be an unwanted anomoly.  
    Fix it and look in the \codeterm{BUGS} file for information
    about how to transmit this knowledge to us.
    Keep in mind that the feature you fix may not appear in
    the next release of the library.
\end{enumerate}


\section{General}
\begin{itemize}
\item
    \module{ProgArgs} upcalls command line arguments multiple times.
    be careful to declare all registered flags in global memory. 
\end{itemize}


\section{I/O}
\begin{itemize}
\item
    \module{TextIO} is built upon \module{BinaryIO}.  Hopefully,
    most implementations will be able to keep this structure, but
    don't depend upon it too heavily.
\item
    \module{ScreenIO} is new and probably buggy.  We have not used it
    in many applications.  
\end{itemize}

\section{Formatted I/O}

\begin{itemize}
\item	
    \module{FIOBase} is inefficient and it may change drastically in
    the future.  If you define your own formats, please be aware of
    this possibility.
\item
    \module{DataFIO} and \module{ConvertFIO} depend upon \term{string files}
    supported by \module{String\-BinaryIO}.  This may not be available on
    all implementations.
\item
    \module{FIOScreen} is not in the default set of formats, because it
    is not necessary for most programs.  You must import it into your
    program somewhere in order to ``activate'' this format.
\end{itemize}


\section{Sun Implementation}

\xquote{Jonathan Swift}{Gulliver's Travels}{
    ``These civil wars were constantly enouraged by the kings of Blefuscu; 
    and when they were quelled, the exiles always fled for refuge to that 
    empire.
    It is computed that eleven thousand persons have at several times suffered 
    death rather than submit to break their eggs at the smaller end.  Now, the
    Big Endian exiles have found so much credit in the Emperor of Blefuscu's
    court, and so much private assistance and encouragement from their private
    party here at home, that a bloody war has been carried on between the two
    empires for six-and-thirty moons, with varied success.''
    }

\subsection{General}
\begin{itemize}
\item
    \module{SafeStorage} uses \proc{malloc} instead of going through
    \module{Storage} (which just calls \proc{malloc}).
\item
    \module{ProgErr} does the core dumps not Sun Modula-2.
    Since \codeterm{core} is dumped by Unix and cannot be done
    by a program which continues executing, the termination code
    is run by \proc{fork}ed process.  There are two advantages:
    the dump cannot be corrupted by termination code and
    the termination code itself can have a bug.  The latter dump
    causes \codeterm{core2} to be written so that you can debug
    both bugs.
\item
    \module{ProgErr} registers with \code{on\_exit}\index{on\_exit}.  If the
    exit status is non-zero, a core dump is generated (unless
    core dumping is turned off via \proc{Set\-Memory\-Save\-Mode}).
\item
    String constants for program arguments and environment variables
    are on the ``short'' side for performance reasons. 
\end{itemize}

\subsection{I/O}
\begin{itemize}
\item
    You should declare your strings of an even size so that 
    \proc{FormatIO} works well.  All strings are rounded up to
    even boundaries, so \proc{FormatIO} may occasionally
    output an odd character or read one character more than
    the number characters in the string
    declaration. 
\item
    The \module{IOVersions} separator is \verb!#! (sharp).
\item
    The \module{IONames} default directory is \code{/tmp}.
    It does not use \proc{mktmp}, because this doesn't provide
    the option of creating temporary files in the current directory.
    
\item
    There are no non-resettable I/O errors unless the internal file
    is not open (\codeterm{IOErrors.\-not\-Open}).
    
\item
    \module{ScreenIO} uses \proc{TextIO.\-Get\-Input} and 
    \proc{TextIO.\-Get\-Output} if they are tty's.  Otherwise,
    the file \code{/dev/\-tty} is used. 

\item
    \module{IOConsts} declares its own values for \code{MAXNAMLEN} and
    \codeterm{MAXPATH}, because these values are too large for Modula-2
    (i.e. using 1024 really degrades the performance of a program if
    it uses file names extensively).
\item
    \proc{UnixBinaryIO.Flush} does a \proc{fsync}, therefore you shouldn't
    call this procedure on disk files unless you really want atomic
    transactions.  If you just want to make sure the data is not buffered
    in \module{TextIO}, use the option \codeterm{TextIO.\-do\-Not\-Buffer}.
\end{itemize}

\subsection{Lightweight Processes}

\begin{itemize}
\item
    \term{Floating point registers} {\em are not saved}; therefore, problems
    may occur.  Putting the code in isn't too complicated if you
    know which chip is to be used.  There are several alternatives
    to modifying the process switching code: use software floating
    point, do all floating point operations in one task, or 
    make the lightweight process system non-preemptable 
    (see \module{TaskConsts}).
\item
    \codeterm{errno} is saved on a per task basis like any other
    register.  Interrupt handlers must save the value of errno if
    doing an operation which might affect \codeterm{errno}.   See
    \module{Interrupts} for more details.
\item
    The implementation of LWP hasn't had much use before this
    release.  Therefore, be wary of subtle errors in the low
    level code.  All of the code in \codeterm{lwp.\-src}
    has been in use for a while.
\item
    \codeterm{SIGSEGV} is never blocked by the LWP system, because
    bad pointer checking is done via this signal.  This doesn't
    prevent you from catching this signal, but be aware that it
    may be in a critical section.  To change this behavior (not
    recommended) modify the values in \module{Tasks\-PRIV\-ATE}
    and \module{Inter\-rupts}.  
\item
    \module{Interrupts} only catches signals if someone is
    registered for them.
\item
    \module{Task\-Time} uses the real time interval timer
    (\codeterm{ITREAL}).
    Use \module{Task\-Time} for timing; don't register for the
    signal.
\item
    \module{Inter\-rupts} catches signals on one stack which
    is its own.  This was done because Unix eats up stack
    space to such a degree that the \codeterm{min\-Stack\-Size} would
    have to have been 1000 bytes or so.
\item    
    \module{UnixAsynchIO} is a little flaky.  All files are
    treated as asynchronous by \module{Unix\-Bin\-aryIO} which
    also means stdin, stdout, and stderr.   If the program
    crashes hard (e.g. segmentation fault), the tty will be
    left in the no-delay state (\codeterm{FNDELAY}).  If a
    process doesn't expect this (e.g. \codeterm{dbx}),
    it will have problems.  The program \codeterm{tty\-reset}
    has been supplied to reset the terminal after hard crashes
    when debugging.  Note also that to catch \codeterm{SIGIO}
    one must be the registered owner of the tty.  This can have
    strange effects if you redirect the output or input of an
    LWP program to another window.

\item
    Stack checking is performed only during task switches, hence the
    amount of maximum stack space used is not as good as it could be.
    
    \finepoint{
	If you are in desperate need of better stack utilization
	statistics, you could run a high priority task that sleeps
	for a short period of time.  This would, in effect, cause
	stack checking to occur more often at the expense of slower
	program execution.
    }
\item
    See the file \code{misc/\-dbx.\-shr} for a description of how to
    debug with Tasks and dbx.
\end{itemize}

\newpage
\section{IBM-PC/MS-DOS Implementation}

\xquote{Jonathan Swift}{Gulliver's Travels}{
    ``These two mighty powers have been engaged in a most obstinate war for
    six-and-thirty moons past.  It began upon the following occasion: 
    it is allowed on all hands that the primitive way of breaking eggs, 
    before we eat them, was upon the larger end;  but his present
    majesty's grandfather, while he was a boy, going to eat an egg and
    breaking it according to ancient practice, happened to cut one
    of his fingers; whereupon the emperor, his father, published an
    edict command all his subjects, upon great penalties, to break
    the smaller end of the eggs.''
    }

As a general rule, you should not access the modules System,
RTSMain, Storage, etc.

\subsection{General}
\begin{itemize}
\item
    \proc{Lists.Adr\-Compare} gets around the Logitech bug that
    you can compare addresses with different segments.  The 
    algorithm is 100\% ugly, but it appears to work.
\item
    \codeterm{Sys\-Types.\-CARD\-INAL32} has been thoroughly
    used and is tightly 
    integrated with the library: that is, \module{FIOCard},
    \module{CardConvert}, and \module{CardIO} all support
    32 bit cardinal reads.
    
\item
    \codeterm{Sys\-Types.\-INT\-EGER32} is not tightly integrated
    because we have \code{CARD\-INAL32}.
    
\item
    \module{ProgArgs} accepts double quote characters when parsing
    the command line.  There is no escape.

\item
    \module{SafeStorage} normally uses the module \module{Storage}
    for all of its allocates.  
    However, if the preprocessor
    variable \codeterm{No\-Stor\-age} is true, the module will
    be compiled to use \term{DOS}.
    
\end{itemize}
\subsection{I/O}
\begin{itemize}
\item
    \module{ProgErrOutput} default output is to {\em stdout} and
    not {\em stderr}.  This is because you can't redirect stderr
    from the command line.  The same goes for 
    \proc{Get\-Error\-Out\-put}.  If you want other semantics, 
    modify the module \module{Dos\-Binary\-IO}.
    
\item
    \proc{Dos\-BinaryIO.\-Flush} closes and re-opens
    the files to make sure the directory information gets updated.
    DOS doesn't have any problems with getting data to the disk.
    However, if the directory isn't updated and the program crashes,
    the data will have to be recovered with one of those magic
    disk utilies.  This implementation of flush causes a bit of
    overhead, because \module{Dos\-BinaryIO}
    must save the absolute file name in the the object record.
    Given that file I/O is so horrendously slow on the PC, this 
    really doesn't make a great deal of difference except that
    the {\em initial default files cannot be flushed}.
    (The default files are input, output, and error output.)
\item
    \module{ScreenIO} does not support edit or echo modes.
    To implement these features, one would need to write a
    buffered character reader for the console device.  Note
    that this would not work if standard input was redirected.
    
    If the standard input or output is redirected, \module{ScreenIO}
    opens the device \codeterm{con:} to make sure the program gets
    to the screen.
    
    The module is designed to work with the ANSI terminal emulator,
    so you must have \codeterm{ansi.sys} installed as a device.  For
    higher performance I/O, you can steal some of the code from the
    \codeterm{demo} directory.
    
\item
    \module{TextIO} interprets a new-line to be:
    \begin{itemize}
    \item
	The sequence \codeterm{ASCII.\-cr}
	followed by an \codeterm{ASCII.\-lf} (DOS standard).
    \item
	An \codeterm{ASCII.\-cr} which is not followed by an
	\codeterm{ASCII.\-lf} (Raw input).
    \item
	An \codeterm{ASCII.\-lf} which is not preceded by an
	\codeterm{ASCII.\-cr} (Unix standard).
    \end{itemize}
\item
    The \module{IOVersions} separator is \verb!#! (sharp).
    
\item
    The \module{IONames} default directory is the current directory.
    It does not use the DOS call for making temporary files since
    it is not available in DOS 2.0 or later.

\item
    \module{ScreenIO} uses \proc{TextIO.\-Get\-Input} and 
    \proc{TextIO.\-Get\-Output} if they are the console device.
    Otherwise, the file \code{con:} is used. 
\item
    \module{Keyboard} uses the \term{BIOS}, so 
    \proc{ScreenIO.\-Get\-Input} may not be the same source 
    (i.e. if the console has been redirected).
\item
    There are two non-resettable I/O error which occurs when \proc{Flush}
    fails in its attempt to reopen the file or when the internal file
    is not open (\codeterm{IOErrors.\-not\-Open}).
    
\end{itemize}

\subsection{Lightweight Processes}
\begin{itemize}
\item
    Don't look at the module \module{Interrupts} unless you have
    a strong stomach.  This implementation works with minimal
    differences between Logitech 2.0 and 3.0 so we kept it.  Perhaps
    if we had Logitech 3.0 to start with\ldots
    
\item	
    The module \module{Interrupts} catches the: 
    \begin{itemize}
    \item
	\term{keyboard interrupt} (\code{09H}),
    \item
	\term{BIOS control-break interrupt} (\code{1BH}),
    \item
	\term{DOS control-break exit interrupt} (\code{23H}), and
    \item
	\term{DOS critical interrupt} (\code{24H}).
    \end{itemize}
    The BIOS turns on interrupts {\em before} it has reset the
    8259 mask which causes deadlock.  The keyboard interrupt
    is caught to put a protective envelope around the BIOS.
    
    The DOS critical error interrupt is caught for similar reasons.
    The normal I/O modules will receive the error and pass it on up.
    The disadvantage is that the user doesn't get the message
    \code{Abort, Retry, or Ignore?}, so the program must handle
    floppy or network drive timeouts.
\item
    \module{TaskTime} catches the hardware clock interrupt (\code{08H})
    and calls the previous handler.
\item
    \term{Floating point registers} {\em are not saved}, therefore problems
    may occur.  Putting in the code is a bit hairy if you want to
    maintain debuggability, therefore applications which do floating
    point and use the 8087 or 8287 should restrict the operations
    to a single task {\em or} make the LWP system non-preemptable
    (see \module{Task\-Consts}).
\end{itemize}

\subsubsection{Logitech 3.X}
\begin{itemize}
\item
    The LWP environment under 3.X runs with both the \term{PMD} and 
    \term{RTD}.  Be careful in the RTD, because it may
    swap out the code of a running task.  You may want to compile
    the program so that it runs in non-preemptable mode when
    you need to debug a LWP application.  See the module 
    \module{Task\-Consts} for information on how to do this.
\item
    To look at a task's call frame, you should {\em E}xamine the 
    field \code{task}\verb*!^!\code{.\-stack\-Base.\-logi\-tech}.  
    Note the
    first task object is located in \codeterm{Tasks\-PRIV\-ATE.\-main\-Task}.
\item
    The value of the lowest stack pointer (\codeterm{lowestSP})
    is not checked unless you assemble your own version of the
    \module{RTSMain} with \codeterm{StackStatistic} turned on.
\end{itemize}

\subsubsection{Logitech 2.X}
\index{Logitech 2.X}
    
    
\begin{itemize}
\item
    The supplied run-time system \codeterm{m2.exe} doesn't work
    with LWP; you will need to use \codeterm{m2lwp.exe}.  The
    same goes for the supplied \codeterm{l2erts.l2e} which is
    replaced by \codeterm{l2erts.lwp}.  For more information
    look in the directory \code{gen}\verb*!\!\-\code{lg2.\-src}.  Note
    also that the implementation of the module
    \module{System} supplied by Logitech is also replaced.
\item
    The \term{RTD} doesn't work with LWP.
\item
    The \term{PMD} does work and you can even get the call frames
    of any \codeterm{Tasks.\-Object}.  To look at a task's call
    chain, you must have access to the task's pointer.  The
    main task's object is: \codeterm{Tasks\-PRIV\-ATE.\-main\-Task}.
    To access its call frame, display the data for the module
    \module{Tasks\-PRIV\-ATE} and look at the field: \codeterm{logitech}.
    This displays a list of saved registers.  Then type:
    \codeterm{=RE}.  This switches to the {\em raw} window
    and then to the {\em call} window of the task.

    {\em IMPORTANT: The library must have been compiled with the
    preprocessor variable \codeterm{Debug} set to true.}
    
\item
    If there is an error in one of the termination procedures
    and \codeterm{memory.pmd} has already been written, a 
    second dump is taken using the name \codeterm{memory.pm2}.
    
\item
    \module{SafeStorage} uses \term{DOS} with LWP under 2.0.

\end{itemize}

\subsection{Debugging}
\begin{itemize}
\item
    All library directories generate three batch files: 
    \codeterm{m2pmd.bat}, \codeterm{m2rtd.bat}, and
    \codeterm{m2env.bat}.   The first two set up the environment
    for the \term{PMD} and \term{RTD} (respectively) and then
    call the programs.  The last script merely sets up the environment
    variables so you can compile and use \codeterm{m2pp}.
\end{itemize}

