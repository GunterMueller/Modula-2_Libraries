% Copyright 1988 by Olsen & Associates (O&A), Zurich, Switzerland.
%
%		    All Rights Reserved
%
%
% Permission to use, copy, modify, and distribute this software and its
% documentation for any purpose and without fee is hereby granted,
% provided that the above copyright notice appear in all copies, and
% that both that copyright notice and this permission notice appear in
% supporting documentation, and that all modifications of this software
% or its documentation not made by O&A or its agents are accompanied
% by a prominent notice stating who made the modifications and the date
% of the modifications.
%
% O&A DISCLAIMS ALL WARRANTIES WITH REGARD TO THIS SOFTWARE AND ITS
% DOCUMENTATION, INCLUDING ALL IMPLIED WARRANTIES OF MERCHANTABILITY AND
% FITNESS.  IN NO EVENT SHALL O&A BE LIABLE FOR ANY SPECIAL, INDIRECT OR
% CONSEQUENTIAL DAMAGES, ANY DAMAGES WHATSOEVER RESULTING FROM LOSS OF
% USE, DATA OR PROFITS, WHETHER IN AN ACTION OF CONTRACT, NEGLIGENCE OR
% OTHER TORTIOUS ACTION, ARISING OUT OF OR IN CONNECTION WITH THE USE OR
% PERFORMANCE OF THIS SOFTWARE OR ITS DOCUMENTATION.
%%%%%%%%%%%%%

\chapter{Errors and Program  Termination}
\label{IOErrorsChapter}
\label{ErrorsChapter}
\xquote{Bill Granger}{There Are No Spies}{
    Devereaux opened the door at the end of the car and dropped from 
the train onto the platform as the electrified express to Geneva
quickly picked up speed.  A conductor at the far end of the platform
frowned at Devereaux.  He walked over and shook his finger and told
him the dangers of jumping from a moving train.  Devereaux had
broken the rules in a country of rules.
}

Modula-2 does not sport many facilities for building robust
programs.  Therefore, error handling in the \library\ is limited.
There are two types of errors treated by the library:
user and program.  A \term{user error} (sometimes called \term{data error})
is caused by an event external to the program.  A \term{program error}
(sometimes called an \term{assertion fault} or {bug}) is 
caused by a logical failure at the time the program was written.
User errors often lead to program errors.  This cascading
effect can cause very poor program behavior which frustrates
users.  For the purposes of this discussion, error detection
is the onus of the programmer and not the user.

Detecting errors is difficult.  Handling them is even more
difficult.  There are four methods of handling errors in the library:
\begin{itemize}
\item   
    Modula-2 type checking facilities are used to handle and
    prevent errors.
\item
    A value is returned by the failed operation.
\item    
    The program is terminated.
\item
    An \newterm{upcall} is executed when the error occurs to 
    allow the program to correct the error or at least to 
    terminate gracefully.
\end{itemize}

\newpage
\finepoint{
    Another method, \term{exceptions}, was thrown out early in the
    design of the library for the following reasons:
    \begin{itemize}
    \item
        Lack of language support would place more pressure on the
        programmer to use exceptions properly.
        Failure to do so would result in situations which
        were very hard to debug.
    \item
        Resources acquired during the calling sequence which led
        to an exception need to be freed to prevent deadlock or
        loss.  Some systems
        solve this problem by integrating a small set of resources
        (e.g. critical section monitors) into the exception handling 
        mechanism.  This solution is insufficient
        in dealing with the general problem of handling
        external devices, memory, and process deadlocks not caused
        by monitors.
    \item
        Other considerations are added implementation complexity (bigger
        basic system) and greater system dependence (reduced portability).
    \end{itemize}
    It would be straightforward introduce a module which supports 
    exceptions and integrate it with the 
    library.  The problem would be changing all the procedures
    (particularly I/O) to not implement error handling via return
    values.
}

This chapter discusses various methods by which errors are detected
and handled in the library.  The general philosophy of the library is
to return user errors and terminate on program errors.  Errors 
are detected as early as possible.
    
\section{\term{Assertion Checking}}\index{\code{Assert}}

Modula-2 is a strongly typed language which means language processors
employ strict type rule checking.  The library employs type checking
as well as the more general system called \newterm{assertion checking}.
An assertion check is the testing of a condition which is supposed to
be true.  If the condition is false, the program is deemed to be in error
and is
terminated at the point the error is detected.  As type checking
can be avoided, so can assertion checking.  The Modula-2 preprocessor
variable \codeterm{Assert} is used to include assertion checks.
If the library is compiled with this variable set to false, no
assertion (or very rudimentary) assertion checking is performed.
A typical usage follows:

\begin{verbatim}
    @IF Assert THEN
        IF ( ( startIndex > lengthSource ) AND ( startIndex # EOS ) ) OR
           ( ( stopIndex > lengthSource ) AND ( stopIndex # EOS ) ) THEN
            ProgErr.Terminate( 'Strings: range error.' );
        END;
    @END
\end{verbatim}

In this example, \code{Strings.\-Del\-ete} is verifying that
its input parameters make sense.  An errant program might pass
uninitialized or incorrect values and the desired result
would not be obtained.  The library assumes this is an error
and terminates the caller via the procedure \code{Prog\-Err.Term\-inate}
just as most Modula-2 language processors would terminate a program
that indexes an array variable outside of the array's defined bounds.
See chapter\ref{ObjectsChapter} for high-level pointer assertion
checking.


\section{Graceful Termination}\index{Termination}

\xquote{Mark Twain}{Life on the Mississippi}{
    ``Well, in ordinary times, a person dies, and we lay him up
    on ice; one day, two days, maybe three, to wait for friends to
    come.  Takes a lot of it --- melts fast.  We charge jewelry 
    rates for that ice, and war prices for attendance. ...
    Same with Embamming.  You take a family that's able to embam, and 
    you've got a soft thing.  You can mention sixteen different ways to 
    do it --- though there ain't only one or two ways, when you come 
    down to the bottom facts of it --- and they'll take the 
    highest-priced way, every time.  It's human nature --- human
    nature in grief.''
}

Software is rarely free of bugs.  In the previous example, the
procedure \proc{Prog\-Err.\-Term\-inate} was called to stop
program execution.  Normally, Modula-2 programs call the
built-in procedure \proc{HALT}.  The advantage of calling the
module \module{ProgErr} is that \module{ProgErr} allows for
\newterm{graceful termination} in the event that it can catch the
error.  \module{Prog\-Err} supports notification of program
termination so that an application can clean up resources
(e.g. clearing the screen, deleting temporary files, etc.)
before finally exiting.  On some implementations \module{Prog\-Err}
can allow the program to catch all termination errors, but
at a minimum it will allow graceful termination when 
\proc{Term\-inate} is called.

\subsection{\module{Notices}}
\label{Notices} 

A notice is a method of inter-module communication as distinguished
from inter-task communication.  Modules register procedures 
with \module{Notices} declared by other modules.  When a notice 
is {\em called}, or invoked, all of the procedures registered
with the notice are called.  A notice may be called multiple
times and registrants may deregister themselves at any time (even
during execution of the notice).  

There are two default \codeterm{Not\-ices.\-Obj\-ect} exported
from \module{Notices}:
termination and out of memory. The latter is discussed in
section~\ref{OutOfMemory}.  Modules register with the termination
notice if they have resources that must be cleaned up before
termination.  The following shows a typical registration:
\begin{verbatim}
Notices.Register( Notices.GetTerminationNotice(), Terminate );
\end{verbatim}
The \proc{Get\-Term\-ina\-tion\-Notice} returns the 
globally defined \term{termination notice} which is
a \codeterm{No\-tices.\-Obj\-ect}.
\code{Term\-inate} is a procedure local to the module which is
declared as follows:
\begin{verbatim}
    PROCEDURE Terminate(
        mode : SysTypes.ANYPTR
        );
        BEGIN (* Terminate *)

            IF mode = Notices.normalTermination THEN
                ...Do normal clean up...
            ELSE
                ...Do quick and dirty clean up...
            END;

        END Terminate;
\end{verbatim}

When any notice is called, there is a single argument passed to
all registered procedures.  The argument is a \code{Sys\-Types.\-ANYPTR}.
In the case of a termination notice, if this value is equal to
\codeterm{No\-tices.\-nor\-mal\-Term\-ina\-tion}, the program
has run its normal course.  If the value (\code{mode} parameter
above) is not \codeterm{\-nor\-mal\-Term\-ina\-tion}, the program
was terminated by one of the following methods:
\begin{itemize}
\item
    \proc{Prog\-Err.\-Term\-inate};
\item
    \proc{HALT};
\item
    Run-time error;
\item
    Another system dependent method of error termination.
\end{itemize}
The only way to terminate a program normally is to return
from the program module's initialization body.  
In all cases, the module \module{Prog\-Err} attempts to provide
graceful termination.

The procedures that are registered with a notice are called in
one of two orders: \codeterm{first\-Regist\-ered\-First\-Called}
or \codeterm{first\-Regist\-ered\-Last\-Called}.
The termination notice is the latter call order which is the compliment
of the initialization body call order defined by Modula-2.

\section{User Errors -- I/O Errors}

Most of the I/O modules allow the program to catch errors caused
by the user.  User error means any error  which the programmer 
did not anticipate.  These errors include:
\begin{itemize}
\item
    Incorrectly entered data or corrupted data files.
\item
    Hardware failures.
\item
    Insufficient resources (e.g. no more disk space).
\item
    File access failures.
\item
    Any other error returned by the host operating system.
\end{itemize}

The module \module{IOErrors} specifies a list of errors 
(\codeterm{IOEr\-rors.\-Er\-rors}) available in all implementations
of the library. Some implementations may also provide access to 
the errors provided by the operating system.  The modules \module{TextIO}
and \module{BinaryIO} return a state which indicates whether an I/O
error has occurred.  If the file is in an \codeterm{error} state,
the procedure \proc{Get\-Error} (exported from both modules)
returns the particular I/O error.

Once an I/O error is detected on an open \codeterm{BinaryIO.\-Object}
or \codeterm{TextIO.\-Object}, the program must either correct the error
or close the file object  (or just terminate the program).  Failure to take
either of these actions will result in {\em termination} if a  subsequent I/O 
operation is attempted.  In other words, either fix the error or do not
use the file again.


There are
two parts to I/O error correction: repairing the external file error and
reseting the \code{error} state.  Repairing the external file error may
be as simple as reprompting the user (e.g. ``\code{Please re-enter: }'')
or it may involve replacing external hardware.  Once the \term{external file}
is repaired (or to simply retry the operation), the procedure
\proc{Set\-Error} (exported by both modules) can be used to reset
the error state of the \term{internal file} as follows:
\begin{verbatim}
    IF TextIO.SetError( file, IOErrors.ok ) # TextIO.error THEN
        ...Error reset, try again...
    ELSE
        ...Unresettable error...
    END;
\end{verbatim}
As demonstrated, \proc{Set\-Er\-ror} can fail, but only in the
case of non-resetable errors (see implementation details for
your particular implementation).  If the error was reset,
the state of the file is the same as it was before the error was
imposed.

\proc{Set\-Error} also can be used to impose errors on the file.
This is useful when building I/O primitives.  An example can
be found in section~\ref{TextIOScan}.


\subsection{\module{SimpleIO} and \module{SimpleFIO} errors}

Most I/O modules return an internal file state which can be
used to detect failure.  The modules \module{SimpleIO} and
\module{SimpleFIO} do not return an error state; they terminate
the program in the event of I/O errors.  There are two special
cases for which the program is allowed to continue:
\begin{itemize}
\item
    If a {\em read} operation fails with the error
    \codeterm{IOEr\-rors.\-bad\-Data}.

\item
    If a {\em write} operation fails with the error
    \codeterm{IOEr\-rors.\-bad\-Para\-meter}. 
\end{itemize}
In the read case, the operation returns \code{FALSE} to indicate
failure.  Write operations do not give any indication of the error;
they just continue.  \codeterm{bad\-Para\-meter} indicates that the
program passed one or more invalid parameters to the write procedure
(e.g. passing a bad format string).  In some cases, the write procedure
will output something (e.g. \code{*****} \`{a} la {\sc FORTRAN}).  

\subsection{Format Errors}

\module{FormatIO} is flexible, but error prone.  Programmers new
to the language of format strings often make subtle syntax or
semantic errors.  To help the programmer, the module \module{ConfIO}
supports verbose diagnostic output.  If an error occurs that would
result in \codeterm{bad\-Para\-meter}, \module{FormatIO} family 
outputs an error message and terminate the program at the fault point.
Termination is necessary, because it provides the programmer more means
to debug the error.

The default for this {\em error mode} is to return \code{bad\-Para\-meter}
to the caller.  To change this default, one may link the program with
a new implementation of the module \module{ConfFIO}.  A new initialization
body is all that is required as shown below:
\begin{verbatim}
    BEGIN (* ConfFIO *)
        errorMode := terminateOnErrors;
    END ConfIO.
\end{verbatim}

One may also change the \codeterm{error\-Mode} at run-time by
calling \proc{ConfFIO.\-Set\-Error\-Mode} with the value
\codeterm{ConfFIO.\-term\-inate\-On\-Er\-rors}.

\subsection{Error Messages}

The modules \module{IOErrors}, \module{BinaryIO}, and \module{TextIO}
export procedures for printing messages to \module{Prog\-Err\-Output}.
The procedures \proc{TextIO.Print\-Error\-Message} and 
\proc{BinaryIO.Print\-Error\-Message} are used as follows:
\begin{verbatim}
    TextIO.PrintErrorMessage( file, '' );
\end{verbatim}
The output of this call might look like: 
\begin{verbatim}
    "somefile", line 36: Bad data in file
\end{verbatim}
If the string passed to \proc{Print\-Er\-ror\-Mes\-sage} is non-null,
e.g. ``{\tt 'open comment'}'', then the output would look like:
\begin{verbatim}
    "somefile", line 36: open comment
\end{verbatim}
The \module{BinaryIO.\-Print\-Er\-ror\-Mes\-ssage} is the same
except that the ``\code{, line 36}'' value is not printed.

\code{IOErrors.PrintMessage} supports printing of strings and
error codes as follows:
\begin{verbatim}
    IOErrors.PrintMessage( IOErrors.invalidIndex, 'hello' );
\end{verbatim}
which outputs:
\begin{verbatim}
    hello: Invalid file index
\end{verbatim}

\section{\module{Safe\-Stor\-age}}
\label{OutOfMemory}

The definition of the module \module{Storage} is to terminate
on errors.  While this is a reasonable for most applications,
some applications require a less drastic alternative.   The
module \module{Safe\-Stor\-age} is provided to help larger
applications deal with the problem of memory shortages.  

The first method (mentioned in section~\ref{Notices}) is the
\proc{Notices.\-Get\-Out\-Of\-Mem\-ory\-Notice}.  This notice
is called when \proc{Safe\-Storage.\-ALLOC\-ATE} is unable to
supply the amount of memory requested.  Registrants of this
notice must free enough memory to allow the \proc{ALLOC\-ATE}
call to succeed.  \proc{ALLOC\-ATE} is called again automatically
after the registered procedures.  If it fails a second time, the 
procedure is killed.  If too little memory is still available
after the execution of the notice, the program is terminated.

The procedure \proc{AllocateOK} returns failure (as opposed to
calling the out of memory notice) when a particular request cannot
be satisfied.  None of the provided library modules use procedure,
because it would mean returning error codes for conditions that are
not likely to occur, i.e. out of memory just doesn't happen very
often.  However, some applications might require this approach
to determine the amount of available memory (e.g. to allocate large
buffers).

\section{Lightweight Processes -- \module{Tasks}}

All of the above error handling philosophies also hold with 
light weight processes.  Interrupt and task threads share
program data.  If a program error is encountered (e.g. an
assertion fault) the lightweight process system assumes the
entire program is in error and the program is terminated.

\finepoint{
    To aid the program, the termination notice (see section~\ref{Notices})
    is called with the value of the \code{Tasks.\-Object} of the
    fault thread.  This information 
    is used by the module \module{Task\-Debug}, for instance, to 
    print the name of the task in error.  More advance debugging
    systems could provide stack traces of the fault thread.  
    }

\subsection{Normal termination}

A non-LWP program terminates by returning from the program module's
initialization body.  An LWP-based program terminates when
{\em all \codeterm{Tasks.\-Object}s exit}.
A task exits by returning from its thread procedure (i.e. the
procedure supplied to \proc{Tasks.\-Create}).  There are a couple of
reasons for this approach:
\begin{itemize}
\item
    Terminating a task in mid-stream could cause disastrous
    results on systems, and on the IBM PC, where the lightweight process 
    environment
    is directly connected to external hardware.
    If a task is terminated in the middle of some critical operation
    (possibly in the middle of polling a device) by another task,
    the program might exit with the hardware left configured in an unstable
    state.
\item
    It is simpler to implement the lightweight process system when
    one can assume that a task will not be terminated in the middle
    of a critical operation.
\end{itemize}

\subsubsection{\proc{Task\-Notices.\-Get\-Exit\-Request}}

Coordinating the termination of all tasks is a difficult task in an
object-oriented lightweight process environment.  For this reason,
the module \module{Task\.Notices} exports a notice via the procedure
\proc{Get\-Exit\-Request}.  This notice can be called by any {\em task}
(but not interrupt handler) when the task determines it is time to exit
the program.  Implementation modules that use tasks or block tasks
to perform their exported function must either register for the exit
request notice or provide a means for importers to unblock the tasks.
(See the implementation of \module{TaskDebug} for an example usage.)

\finepoint{
    The library provides alternatives for most things, even normal
    task termination.  In other words, the library supplies the 
    rope to all who are willing.  The procedure 
    \proc{Tasks.\-Exit\-System} can be used to terminate an
    LWP program {\em without} requiring that all tasks exit normally.
    See the definition module of \module{Tasks} for the appropriate
    caveats.
}
    
\subsection{\proc{Tasks.\-Panic}}

There are cases when the execution of termination procedures 
(procedures registered with the termination notice) would be
impossible because the lightweight process system is in an
unstable state.  The procedure \proc{Panic} is provided as an
alternative to \proc{Prog\-Err.\-Term\-inate}.  There are
two primary differences between \proc{Panic} and \proc{Terminate}:
\begin{itemize}
\item
    The error message is printed directly to the host operating
    system, i.e. \module{Prog\-Err\-Output} may not be used.
\item
    The termination notice will not be called.
\end{itemize}

\finepoint{
    A program may install a \proc{Panic\-Handler} to clean up the most
    critical resources (such as specially configured external hardware)
    or to cause a {\em reboot} of the system.  See the definition
    module for \module{Tasks} for more details.
    }

\subsection{Other Errors}

Some of the task primitives require that tasks exit when they are
quiescent.  A task is quiescent if it is not holding system resources
that only it can free.  \module{Task\-Monit\-ors} specifies that
a task may not exit if it is {\em holding} a monitor.  
\module{Task\-Mes\-sages} checks to see that a task is not exiting
with pending messages.  These assertion checks are performed during
the deallocation of the \module{Task\-Info} data for the tasks.

\subsection{\module{Task\-Debug}}

The myriad errors that can occur an LWP program are often mind boggling.
To aid the programmer, the module \module{Task\-Debug} is supplied. 
\module{Task\-Debug} supports:
\begin{itemize}
\item
    Catching of user interrupts to display tasking information or
    cause termination.
\item
    Display of task state in the event of abnormal termination.
\item
    Printing of the task state upon request.
\end{itemize}

All the facilities are documented in the definition module
except the display of the procedure \proc{Task\-Debug.Print\-All}.
A call to \proc{Print\-All} results in a display of the form:
\begin{verbatim}
          ***** Current Task State Information *****
Main   State: suspended  Priority: 123/2  Stack: 4/7986
        Sleep: 37821
        Semaphore: Done
        Send: none.  Receives: none.
        Monitors: none.
Producer3B   State: suspended  Priority: 148/1  Stack: 42/4100
        Sleep: 0
        Semaphore: none.
        Send: none.  Receives: Consumer2
        Monitors: <Buffer>
Consumer2   State: suspended  Priority: 148/1  Stack: 42/4108
        Sleep: 0
        Semaphore: none.
        Send: Mon3B  Receives: none.
        Monitors: Buffer[1]
TaskDebugC   State: ready  Priority: 255/2  Stack: 42/4102
        Sleep: 0
        Semaphore: none.
        Send: none.  Receives: none.
        Monitors: none.
IdleFF   State: ready  Priority: 0/0  Stack: 42/270
        Sleep: 0
        Semaphore: none.
        Send: none.  Receives: none.
        Monitors: none.
DeathFD   State: suspended  Priority: 255/1  Stack: 42/388
        Sleep: 0
        Semaphore: none.
        Send: none.  Receives: none.
        Monitors: none.
\end{verbatim}

\finepoint{
    We have included a trivial deadlock situation in this 
    listing.  \code{Producer3B} is waiting for the monitor
    \code{Buffer}.  \code{Consumer2} holds this monitor.
    To complete the loop, \code{Consumer2} is sending a
    message to \code{Producer3B}.  (This would be more obvious
    if \module{TaskDebug} printed pictures instead of listings.)
}

The number of lines of information will vary depending upon which
tasking primitives have been imported.  This program imported all
of the tasking primitives.  Decomposing the state of task 
\code{Prod\-ucer3B}, we have:
\begin{description}
\item[\code{Producer3B}] 
    is the name of the task.
\item[\code{State: suspended}] 
    is the state of the task. The other possible value is \code{ready}.
\item[\code{Priority: 148/1}] 
    is the task priority/preemption level.  Task priority is the
    value obtained from \proc{Tasks.\-Get\-Priority}.  The preemption
    level is the number of calls to \proc{Tasks.Dis\-allow\-Pre\-empt\-ion}
    have made without a corresponding \proc{Tasks.Allow\-Pre\-empt\-ion}.
    In this case, the priority is ``\code{148}'' and the preemption level
    is ``\code{1}''.
\item[\code{Stack: 42/4100}] 
    is the maximum amount of stack consumed followed by the actual size of
    the stack.  The amount of stack consumed may not be accurate on all
    systems.
\item[\code{Sleep: 0}]
    is the amount of the task has left to sleep in milliseconds.
\item[\code{Semaphore: none.}]
    is the name of the \codeterm{Task\-Sema\-phores.\-Object} on which
    the task is blocked.  In this case, the task is not blocked on a 
    semaphore.
\item[\code{Send: none. Receives: Consumer2}]
    is the name of the task to which this task is sending and the
    receive queue for this task.  \code{Producer3B} is not sending
    to anyone, but \code{Consumer32} is blocked sending to \code{Producer3B}.
    If the task has {\em received} but has yet to {\em reply} from a
    task in the queue, the task will be enclosed in angle brackets,
    e.g. \code{<Consumer32>}.
\item[Monitors: $<$Buffer$>$]
    shows on which monitors the task is blocked and which monitors
    are held by the task.  \code{Producer3B} is blocked on the monitor
    named \code{Buffer}.  If it had been holding the monitor, the output
    would look like: \code{Buffer[1]}.  The number in the square brackets
    indicates the nesting level, i.e. the number of times the monitor
    has been {\em entered}.
\end{description}

\finepoint{
    Task names are augmented on some systems with a 
    key to ensure uniqueness in \module{Task\-Debug} outputs.
    You may disable this feature by compiling \module{Tasks}
    with the preprocessor variable \codeterm{Debug} set to false.
    }
    


