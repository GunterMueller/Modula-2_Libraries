% Copyright 1988 by Olsen & Associates (O&A), Zurich, Switzerland.
%
%		    All Rights Reserved
%
%
% Permission to use, copy, modify, and distribute this software and its
% documentation for any purpose and without fee is hereby granted,
% provided that the above copyright notice appear in all copies, and
% that both that copyright notice and this permission notice appear in
% supporting documentation, and that all modifications of this software
% or its documentation not made by O&A or its agents are accompanied
% by a prominent notice stating who made the modifications and the date
% of the modifications.
%
% O&A DISCLAIMS ALL WARRANTIES WITH REGARD TO THIS SOFTWARE AND ITS
% DOCUMENTATION, INCLUDING ALL IMPLIED WARRANTIES OF MERCHANTABILITY AND
% FITNESS.  IN NO EVENT SHALL O&A BE LIABLE FOR ANY SPECIAL, INDIRECT OR
% CONSEQUENTIAL DAMAGES, ANY DAMAGES WHATSOEVER RESULTING FROM LOSS OF
% USE, DATA OR PROFITS, WHETHER IN AN ACTION OF CONTRACT, NEGLIGENCE OR
% OTHER TORTIOUS ACTION, ARISING OUT OF OR IN CONNECTION WITH THE USE OR
% PERFORMANCE OF THIS SOFTWARE OR ITS DOCUMENTATION.
%%%%%%%%%%%%%

\chapter{Installation}




\smallskip
{\bf 
{\centering{Copyright 1988 by Olsen \& Associates (O\&A), Z\"urich, 
    Switzerland.\\
  \smallskip
    All Rights Reserved.}
\smallskip

\begin{quote}
Permission to use, copy, modify, and distribute this software and its
documentation for any purpose and without fee is hereby granted,
provided that the above copyright notice appear in all copies, and
that both that copyright notice and this permission notice appear in
supporting documentation, and that all modifications of this software
or its documentation not made by O\&A or its agents are accompanied
by a prominent notice stating who made the modifications and the date
of the modifications.
\end{quote}
\smallskip
}
}

{\em
\begin{quote}
O\&A DISCLAIMS ALL WARRANTIES WITH REGARD TO THIS SOFTWARE AND ITS
DOCUMENTATION, INCLUDING ALL IMPLIED WARRANTIES OF MERCHANTABILITY AND
FITNESS.  IN NO EVENT SHALL O\&A BE LIABLE FOR ANY SPECIAL, INDIRECT OR
CONSEQUENTIAL DAMAGES, ANY DAMAGES WHATSOEVER RESULTING FROM LOSS OF
USE, DATA OR PROFITS, WHETHER IN AN ACTION OF CONTRACT, NEGLIGENCE OR
OTHER TORTIOUS ACTION, ARISING OUT OF OR IN CONNECTION WITH THE USE OR
PERFORMANCE OF THIS SOFTWARE OR ITS DOCUMENTATION.
\end{quote}
}








\section{Requirements}

The \Library\ consumes five megabytes of disk space in its 
source form.
When it is pre-processed and compiled, it takes nearly double that amount,
as
the preprocessed sources are nearly identical in size to the unprocessed
sources.\footnote{There is an option to strip comments when 
	preprocessing which
	cuts out a considerable amount of overhead but makes 
	it slightly inconvenient 
	to debug.  On the PC, with the strip option you need 7.5 megabytes
	to compile and install the library in one go.}
To be safe, 10 megabytes of disk space should be allocated to the task
of compiling the library.  You can get by with less space, but it means
compiling and installing the library in stages.  In either case, you must select
a home directory for sources of the library and the installed version.
Once the library has been installed, it consumes 1.5
megabytes.

You must be familiar with your particular compiler installation. 
Specifically, you should know where its libraries and, in certain cases, 
the executables are located.  

You should be familiar with
the program called {\em make}.  If you don't know about this program,
consult a Unix manual.  The library comes with
a public domain version of make written in C.  If you are using
a PC, you will find an executable version in the distribution.

\important{ 
The installation instructions and procedures are
provided to help the programmer install the library and tools.
Given that this is not an official product, be very careful
to read the installation scripts before executing them.  This
may save you time and energy during the installation process.
Remember, you get what you pay for!
}
\section{Outline}
The process of installing the library is outlined below:

\begin{enumerate}
\item
    Get files from distribution media to disk in host operating system
    format.
\item
    Run an installation program to distribute the files in the proper
    configuration for the particular language processor.  At this stage,
    files not required by the particular language
    processor used may be trimmed away.
\item
    Set up the compiling environment.
\item
    Compile the library.  It is a good idea to also compile a few 
    test programs to make sure
    that everything is ok.
\item
    Install the library in a public place.
\item
    Try the installed version.
    
\end{enumerate}

\section{File and Directory Structure}
\label{FileStructure}

The library source distribution consists of many files and directories.
This setup makes it easy for us to develop and to distribute the library.
In this section, we describe the structure of the directories and some
of the files (including makefiles).

\subsection{Keywords}
\label{FileKeywords}

There is a single source distribution for all versions of the library.
An install procedure is provided to configure the library directories
for a particular language processor.   The files or directories that are 
language
processor or host operating system specific are identified by a 
code word in their name.  This keyword may appear directly in the file name
or in the name of the directory in which the file lives.  The current 
list of keywords follows:
\index{\code{sun}} \index{\code{pc}} \index{\code{lg2}} \index{\code{lg3}}

\medskip
{\centering
\begin{tabular}{|l|l|}
    \hline

    \code{sun}	   &Sun Modula-2, Sun Operating System, Sun Architecture 
			\\ \hline
    \code{pc}	   &IBM PC Architecure and/or MS-DOS
			\\ \hline
    \code{lg2}	   &Logitech-2/86 Version 2.X
			\\ \hline
    \code{lg3}	   &Logitech-2/86 Version 3.X
			\\ \hline
\end{tabular}
}

\subsection{First Level Directories}


\label{FirstLevelDirectories}

The directories are at most two levels deep.  The top level directories are:

\medskip
{\centering
\begin{tabular}{|l|l|}
    \hline

    \code{bin}		& Default installed home for tools  	
			\\ \hline
    \code{conf}		& Special files for installation 		
				    (it doesn't follow all the rules)
			\\ \hline
    \code{demo}		& Stuff which makes the library look good 
			\\ \hline
    \code{examples}	& Stuff to show that everything isn't so obvious 
			\\ \hline
    \code{doc}		& \LaTeX\ and Unix ``man'' page documentation 	
			\\ \hline
    \code{fio}		& Formatted I/O modules  
			\\ \hline
    \code{gen}		& General library modules
			\\ \hline
    \code{io}		& Fundamental I/O modules
			\\ \hline
    \code{lib}		& Default installed home for the library 
			\\ \hline
    \code{lwp.lib}	& Ditto for lightweight process library 
			\\ \hline
    \code{misc}		& Possibly useful information  
			\\ \hline
    \code{porting}	& Help to port the library
			\\ \hline
    \code{tools}	& Tools used by the library (e.g. preprocessor) 
			\\ \hline
\end{tabular}
}

\subsection{Second Level Directories}
\label{SecondLevelDirectories}

The second level of directories is designed to separate modules with
like names but different implementations.  It also reduces the number
of files in the compilation directories.  One or more of the following
directories exists at the second level depending upon the particular
requirements of the first level directory.  Note that the appearance of
\code{XXX} in this list should be replaced with one of the implementation
specific keywords (e.g. \code{sun}, \code{pc}) described in 
section~\ref{FileKeywords}.

%RJN - is this necessary?
\index{\code{shared.src}} \index{\code{pc.src}} \index{\code{lg2.src}}
\index{\code{lg3.src}} \index{\code{sun.src}} \index{\code{lwp.src}}
\index{\code{pc\_lwp.src}} \index{\code{sun\_lwp.src}} \index{\code{test.src}}
\index{\code{test\_sun.src}} \index{\code{test\_pc.src}} 
\index{\code{test\_sun\_lwp.src}} \index{\code{test\_pc\_lwp.src}}
\index{\code{bin}} \index{\code{lwp.bin}} \index{\code{test.bin}}
\index{\code{test.bin}}

\medskip
{\centering
\begin{tabular}{|l|l|}
    \hline
    \code{shared.src}	    	& Source common to many implementations  
			\\ \hline
    \code{XXX.src}		& Implementation specific source  
			\\ \hline
    \code{lwp.src}		& Common to many implementations 
				  of lightweight processes  
			\\ \hline
    \code{XXX\_lwp.src}		& Implementation specific LWP source 
			\\ \hline
    \code{test.src}		& Source for test programs  
			\\ \hline
    \code{test\_XXX.src}	& Implementation specific tests
			\\ \hline
    \code{test\_XXX\_lwp.src}	& Guess  
			\\ \hline
    \code{bin}			& Compilation directory for non-LWP binaries
			\\ \hline
    \code{lwp.bin}		& Compilation directory for LWP binaries  
			\\ \hline
    \code{test.bin}		& Test program compilation directory 
			\\ \hline
    \code{test\_lwp.bin}	& LWP test program compilation directories
			\\ \hline
\end{tabular}
}

\medskip
On the PC, some of the \code{test} directory names have been
compressed to fit in 8 characters.

\subsection{Make File Structure}
\index{Make} \index{Makefile} \index{m2depend}
\label{Make}

The \code{bin} directories contain many makefiles.  The structure
was designed to be centrally configurable and to allow for automatic
dependency generation.  The purpose
of these individual makefiles is described below.  Standard
Sun and System V {\em make} has an include file facility which we have implemented
in the public domain make supplied on this disk.  If your own implementation
doesn't have include files, use this version.

\begin{description}

\item[\code{makefile}]\index{\code{makefile}}
    is the main makefile.  It specifies the traditional targets:
    \code{all}, \code{install}, \code{clean}, and \code{depend}.
    All \code{bin} directories have identical versions of this file.
    The \code{makefile} includes \code{environ.mak} at its top
    and \code{depend.mak} at its bottom.

\item[\code{environ.mak}]\index{\code{environ.mak}}
    describes the non-automatically generated target lists and
    unique features of the particular \code{bin} directory.  This list is
    kept to a minimum.  The \code{environ.mak} file includes
    two other files at its top: \code{rules.mak} and \code{m2source.mak}.
    
\item[\code{rules.mak}]\index{\code{rules.mak}}
    is located in the installed tools directory.
    This file contains as many of the implementation peculiarities
    as possible.
    
\item[\code{m2source.mak}]\index{\code{m2source.mak}}
    is the list of {\em preprocessed} source files to be compiled.
    It defines the three {\em make} macros: \code{M2Defs}, \code{M2Imps},
    and \code{M2Mods}.  These macros are transformed in \code{rules.\-mak}
    (using special {\em make} macro magic) into the {\em tar\-gets} for this 
    particular directory.
    
\item[\code{depend.mak}]\index{\code{depend.mak}}
    lists the hand-made target dependencies.  These dependencies
    typically tell which source file is to be preprocessed into which
    preprocessed file, e.g.
\begin{verbatim}
	SysTypes.def: ../sun.src/SysTypes.dpp
		${StandardM2pp}
\end{verbatim}
    This file includes \code{m2depend.\-mak}.

\item[\code{m2depend.mak}]\index{\code{m2depend.\-mak}}
    is an automatically generated dependency list for the modules
    specified in \code{m2source.\-mak}.  It also contains some
    {\em make rules} if these are required.
\end{description}
\medskip

Before the installation procedure is executed, there are N copies of each 
of these files (except \code{rules.mak}) in each \code{bin} directory.  
The files for a particular
implementation have a special keyword affixed to identify the
file (see section \ref{FileKeywords}).  After the installation
procedure, the files are named as defined above.  \code{rules.mak}
for each of the implementations is located in the directory \code{conf}.
For examples of all the files listed above, look in the \code{conf}
directory for your particular implementation.  They will also be
copied to the main \code{bin} directory after the installation
procedure.


In the root (home) directory of the library, you will find either
a \code{makefile} or a batch file called \code{make}.  This file
contains the correct compilation order for the directories listed
in section \ref{FirstLevelDirectories} and \ref{SecondLevelDirectories}.

\section{Getting It On Disk}

\subsection{Sun from Tape}
\label{SunCopy}

The library is distributed in {\em tar} format.  For further information,
type: \code{man tar}.  The typical home for the library is
something like: \code{/usr/\-local/\-yaml}.   If you don't like this
location, pick another.  We will use it in the example, however.
Put the tape in the tape drive and type:

\begin{verbatim}
% mkdir /usr/local/yaml
% cd /usr/local/yaml
% tar xvp
\end{verbatim}

If this doesn't work, you may need to specify a particular tape
\code{/dev} name.  
If it works, proceed to section \ref{FileStructure}.

\subsection{PC from Floppy}
\label{PCCopy}

The distribution comes in a series of numbered floppies.  There is a
batch file called \code{xfer} on the first disk which is used to 
get the files from the floppies to the disk.  The home directory of the
library must be the current directory on the destination disk.  Typically,
this is \code{\\yaml}, but it can be anything.  Put the first disk
in the drive and type:
\begin{verbatim}
C> mkdir \yaml
C> cd \yaml
C> a:
A> xfer a c
\end{verbatim}

The script will prompt you for floppies.  Do what it says.  If it doesn't
work, read the script.  The files on the floppies are structured exactly
as they would be on the disk, so you can do this process by hand if necessary.

The script should end with the message \code{batch file missing} and the current
drive should be the hard disk.
If all went well, proceed to section \ref{FileStructure}.

\subsection{Sun from Floppy}
If you get a floppy distribution and want to transfer it to a Sun,
the installation procedure in section \ref{PCCopy} 
can be used to get the files from
floppies to a PC.  If you have PC-NFS, this can be transferred directly to an
NFS partition to which you have access.  Once the files are on the
Sun, type:

\begin{verbatim}
% cd /usr/local/yaml
% sh conf/distrib/dos\_unix.sh
\end{verbatim}
Note: (\code{/usr/local/yaml} is the presumed home
of the library; see section \ref{SunCopy})

\section{Installation Procedure}
\label{InstallProcedure}
\index{\code{lg2\_inst.bat}}\index{\code{lg3\_inst.bat}}
\index{\code{sun\_inst.sh}}

Each language processor has its own installation procedure which can be
found in the directory \code{conf/\-distrib} 
or \code{conf$\backslash$\-distrib}.
You should identify your implementation keyword 
(see section \ref{FileKeywords})to locate a script or batch file
named \code{XXX\_inst}.  Copy this batch file to the home directory
of the library and execute it.  This will move files around and
create some new directories for installing the library and tools.

For Sun,

\begin{verbatim}
% cd /usr/local/yaml
% sh conf/distrib/sun_inst.sh
\end{verbatim}

For Logitech 2.0,
\begin{verbatim}
C> cd \yaml
C> conf\distrib\lg2_inst
\end{verbatim}

For Logitech 3.0,
\begin{verbatim}
C> cd \yaml
C> conf\distrib\lg3_inst
\end{verbatim}

If this script prints any errors, figure out what's wrong and fix it.

\subsection{Optional Part}
\index{\code{lg2\_only.bat}}\index{\code{lg3\_only.bat}}
\index{\code{sun\_only.sh}}


A trimming script or batch file for each implementation is located
in the same directory as \code{XXX\_inst}.
This script deletes files and directories not required for the particular
implementation.  It is named \code{XXX\_only} where \code{XXX} is the
same keyword as used in the installation script.  Copy this script
and execute it in the root directory of the library.  

\newpage
For Sun,
\begin{verbatim}
% cd /usr/local/yaml
% sh conf/distrib/sun_only.sh
\end{verbatim}

For Logitech 2.0,
\begin{verbatim}
C> cd \yaml
C> conf\distrib\lg2_only
\end{verbatim}

For Logitech 3.0,
\begin{verbatim}
C> cd \yaml
C> conf\distrib\lg3_only
\end{verbatim}

\porting{
You probably don't want to delete anything.  
It is useful to have as many examples as possible when figuring 
out what to do.
}

\section{Configuring}
\index{Environment} \index{Configuration}
\index{\code{Root}}

If the installation procedure went according to plan, the
\code{bin} directory in the root directory of the library should
contain several files.  The only file you should need to configure
is \code{rules.mak} (see section \ref{Make}).
This is the main configuration file.  Unless you like
adventure, restrict the changes to this file to the top sections.
The lower in the file you have to go, the deeper in the myre.\footnote{
    Remember, you can always get a fresh copy in 
    \code{conf/\-XXX/\-rules.mak}.
    }

The primary variable to configure is \code{Root}, which identifies
the absolute location of the library.  If you change this variable,
there should be no external effects.  The other variables are 
commented sufficiently so that you should be able to figure them out.  

\portings{ 
    You should look at the files in the \code{porting} 
    directory at this point.

    You may want to add \code{SysTypesCheck} to the list
    of true preprocessor variables.  See the make macro
    \code{M2PPVARS} or \code{M2ppFlags}.
}
\newpage
\msdoss{ 
    MS-DOS names must be kept short.  Therefore,
    you should not change the \code{Root} to be very long.   This problem
    is particularly acute on Logitech 3.0 which requires absolute
    directories in the environment variable \code{M2OBJ}.
}

\index{\code{Strip} Comments}
    You may want to modify the make macro
    \code{M2pp} in \code{rules.mak} to be:
\begin{verbatim}
    M2pp = m2pp/Strip
\end{verbatim}
\begin{quote}
    This will
    save a megabyte or two of disk space.  
    
    {\em WARNING: DO NOT REDISTRIBUTE
    STRIPPED SOURCES; YOU WILL BE IN VIOLATION OF THE COPYRIGHT.}
\end{quote}   

\subsection{\code{PATH} Environment Variable}
\index{\code{PATH} Environment Variable}

In the root \code{bin} directory, you will find a compiled version of
the Modula-2 preprocessor as well as \code{make} if you are running
on the PC.  For the compile to work, you will need to add this directory
to your \code{PATH} environment variable.  For further information on the
\code{PATH} environment variable,
see the manual for your particular
command interpreter.

\section{Compiling}

If the configuration is correct, you should go to the root
directory of the library and type: \code{make}.  However, a more 
prudent approach is to go to the directory \code{gen/\-bin}
or \code{gen$\backslash$\-bin} and execute \code{make -n}.  This prints
out interesting error messages if you have done something wrong.
Otherwise, it lists a myriad of preprocessing and 
compiling commands.

Here is a quick trouble shooting guide:
\begin{description} \index{\code{make} errors}
\item[{\tt Command not found}] ---
    Your \code{PATH} is set incorrectly.

\item[{\tt rules.mak: include file not found}] ---
    The root \code{bin} directory is not in the same place as the
    makefile expects.  Edit {\em all} the \code{environ.mak} files.

\item[{\tt Don't know how to make.}] ---
    This may mean that all the files didn't get on the disk or that
    you have removed some accidently.  The name printed by make should
    give you a good clue as to the problem.  If this is not the case,
    perhaps you should read up on make...
    
\item[{\tt xxx: is up to date}] ---
    Check the clock on your computer.  If it is wrong, you will
    need to set it and start again.

\item[Other messages] ---
    Your guess is as good as mine.

\end{description}

\subsection{Testing}

When you type \code{make} in the root directory of the library,
nothing will be compiled in the \code{test} directories.  This
is to conserve disk space (there were 76 test programs at
last count).  To test the library, you can go into any one of
the test directories and type \code{make}.  
A more appropriate method is to look at some of the test programs and
compile those which you find to be interesting.  The
test programs also serve as poorly written examples of the library.

The \code{demo} directory has a program called \code{Solver}, which
is a multi-tasking automatic maze solver.  It is an unintelligent algorithm
but works the lightweight process environment thoroughly.  If this program runs,
most other programs will probably run as well.


\section{Installing the Library}
\index{Installing} \index{\code{bin} directory} \index{\code{lib} directory}
\index{\code{lwp.lib} directory}

By now you should be a \code{make} expert and possibly have a little
less hair.  Some of the test programs should have been compiled,
run, and work to your satisfaction.  It is time to install the library.
Do this by typing: \code{make install} in the root directory.  This will run a
script or batch file called \code{mk\_install} or \code{mk\_inst} in each
of the compilation directories.  The script copies the binaries to the 
install directories listed in {\tt rules.mak}.

The makefiles have been set up to install files in three locations.
The default names are listed below.

\begin{description}
\item[{\tt bin}] is the installed home of the executable files from 
	the \code{tools} directory.
	
\item[{\tt lib}] is the home of the non-LWP version of the library.

\item[{\tt lwp.lib}] is the home of the LWP binaries.

\end{description}

\code{lib} and \code{lwp.lib} have been separated for the following reasons:
\begin{itemize}
\item
    Lightweight processes are not required for the majority of programs.
\item
    Some modules have different implementations for LWP and non-LWP
    worlds, thus the separation is necessary to avoid naming confusion.
\end{itemize}

\index{Linking Lightweight Processes}
\index{Lightweight Processes, Linking}
If you would like to use LWPs, put the \code{lwp.lib} directory {\em before}
the \code{lib} directory in your module search path.  
This tells the compiler and linker to use the LWP versions
of the modules.  Doing the reverse will cause strange bugs at unpredictable
times.  Look in any of the \code{lwp.bin} directories in the library for
examples of how to set up your path.

\section{Final Test}

After installing the library, try to compile a test program
in your normal working directory.
If all goes well, you may want to archive
the sources and non-installed binaries to free up the disk space.


