% Copyright 1988 by Olsen & Associates (O&A), Zurich, Switzerland.
%
%		    All Rights Reserved
%
%
% Permission to use, copy, modify, and distribute this software and its
% documentation for any purpose and without fee is hereby granted,
% provided that the above copyright notice appear in all copies, and
% that both that copyright notice and this permission notice appear in
% supporting documentation, and that all modifications of this software
% or its documentation not made by O&A or its agents are accompanied
% by a prominent notice stating who made the modifications and the date
% of the modifications.
%
% O&A DISCLAIMS ALL WARRANTIES WITH REGARD TO THIS SOFTWARE AND ITS
% DOCUMENTATION, INCLUDING ALL IMPLIED WARRANTIES OF MERCHANTABILITY AND
% FITNESS.  IN NO EVENT SHALL O&A BE LIABLE FOR ANY SPECIAL, INDIRECT OR
% CONSEQUENTIAL DAMAGES, ANY DAMAGES WHATSOEVER RESULTING FROM LOSS OF
% USE, DATA OR PROFITS, WHETHER IN AN ACTION OF CONTRACT, NEGLIGENCE OR
% OTHER TORTIOUS ACTION, ARISING OUT OF OR IN CONNECTION WITH THE USE OR
% PERFORMANCE OF THIS SOFTWARE OR ITS DOCUMENTATION.
%%%%%%%%%%%%%

\chapter{Tasks Design}

\section{Introduction}
The OIS Display Program is a complex real-time program which 
responds to many asynchronous events, e.g. keyboard input, timers,
serial communication interrupts.  Lightweight Processes (LWPs)\index{LWP}
\index{Lightweight Processes}
have been chosen as the medium to manage these asynchronous events
on the IBM PC.  This paper discusses the implementation of LWPs
and assumes that the reader is familiar with the reasons this
solution was chosen. 

\section{Philosophy}
LWPs are a method of managing events in a system.
\footnote{
    The term \newterm{system} will be used in this document to describe
    a set of sequential, concurrently executing, control threads.  
    This bears no relation to the Olsen Information System other 
    than in the spelling of the word.}
There are two types of \newterm{events} to be managed: 
{\em internal}\index{internal events}\index{events, internal} and 
{\em external}\index{external events}\index{events, external}.
External events correspond to inputs, e.g.
a keystroke, character into the serial port, transmitter ready signal, etc.
They are asynchronous to normally executing control threads,
although some may be regular in nature, e.g. timer interrupts.
Internal events are synchronization points in the system.  These
events cause a temporary connection between two control threads.
All internal events are initiated by the importer of the system
whereas external events may be initiated by either the hardware or the
importer.

The LWP modules (if used properly) provide a strict boundary between
external events and the rest of the sytem.  All external
events are translated to internal events as soon as the conditions
which cause the external events are satisfied.
The theory is that internal events
can be more easily structured (programmed, so to speak) than external events.

In general, the strict model presented is necessary to control real-time
events on the IBM PC.  The IBM PC is a {\em very} fragile environment, 
because memory protection is unavailable.  Combined with a non-reentrant
file system monitor (PC-DOS), the system must be tightly controlled in
order to assure proper resource sharing.  These factors and others 
may cause the design to appear facist.  Be that as it may, this design
assumes the IBM PC is fragile and attempts to protect the programmer
from ``getting bitten'' too often.

\section{General Design}
There are two types of control threads: \newterm{Tasks} and 
\newterm{Interrupt Handlers}.  
Tasks can wait for internal events and cause both types of
events.  Interrupt handlers can wait for external events and cause both
types of events.  The normal mode of operation is for tasks to cause
and wait for internal events and interrupt handlers to wait for external
events and translate them into internal events.  External events are
rarely caused by either tasks or interrupt handlers.

All control threads appear to the importer as a Modula-2 procedure
plus state information (managed by the LWP system).  A task's state information
is incarnated only once during its lifespan.  An interrupt handler's 
internal state information is reinitialized each time an external 
event occurs.  Both tasks and interrupt handlers may access global memory.
Thus, although the internal state information is reinitialized for each
external event, global data may be maintained by the importer to manage
sequences of external events, e.g. an input buffer,
keyboard shift-lock position, etc.

The rest of this paper discusses the relationship between tasks and
interrupt handlers.  The rules about what interrupt handlers can and
cannot do are much simpler than for tasks, thus most of the paper
discusses the inter-relationships between tasks.

\subsection{Tasks}

The basic entity in this model is a \newterm{Task}.  A
task provides an independent control thread for Modula-2 code as well as
PC-DOS function calls.
Tasks are prioritized on a fixed linear
scale which is specified at their creation time.  
Tasks can be in several states:
\begin{itemize}
\item
    Running.
\item
    Ready.
\item
    Suspended.
\end{itemize}


Only one task in the system may be in the \newterm{running} 
state: the task which is currently executing.  A task which is
\newterm{ready} is capable to run, but it is not of higher priority than the
currently running task.
A task may be ``readied'' (awakened) from a suspended
state by another task (or by an interrupt thread).

\subsection{Scheduling}
\label{sec-sched}

When a task is readied, it is put in the ready state.  If it has 
\index{priority}
{\em higher priority} than the running task, then the newly readied
task {\em preempts} the running task; the running task is made
\index{preemption}
ready and the newly readied task is set running.
Preemption is based solely on the priority of the tasks.
The fundamental rule to follow is
that task \code{A} may preempt task \code{B} only if \code{A} is of strictly 
higher priority than \code{B}.  
\footnote{There is an exception to this rule if alternative methods of
scheduling are employed in the system.  For the time being, the rule
is true.}

Tasks can \newterm{allow} and \newterm{disallow} preemption.
If preemption is disallowed, then the currently running task can not
be interrupted by another task (by any means of control). 
If a task disallows preemption, the task must allow preemption at some later
time or no other task will get a chance to run.
This feature provides mutual exclusion over ``very critical'' sections 
of code.  Two uses are prevalent: to notify the LWP system that an
interrupt handler is executing and to build higher level inter-task 
communication primitives.

\subsection{Interrupts}
\index{Interrupts}

Interrupt handlers are not tasks, but are Modula-2 procedures which
are invoked each time a \newterm{vectored interrupt}\footnote{
    There are two types of vectored interrupts on the IBM PC: 
    {\em hardware}\index{hardware interrupts}
    and {\em software}\index{software interrupts}.  The LWP system makes 
    no distinction between the two except that tasks can cause 
    software interrupts, thus they appear as external events
    to the system.} 
occurs.
An interrupt handler's first responsibility is to manage external conditions 
associated with the external event, such as reading characters from 
a UART.  Secondly, it translates
the external event to an internal one, if necessary.  The \proc{Ready}
primitive (section \ref{sec-ctrl}) is the only way
for an interrupt handler to perform this translation.  

An interrupt manager is usually a module which contains the interrupt
handler and provides entry points for tasks to access the data associated
with the interrupt.  Tasks call the interrupt manager and are either
passed the data or suspended until the data is received via an
interrupt.  The exact sharing mechanism is not important, because 
it can be implemented in many ways given the primitives in 
sections \ref{sec-basic} and \ref{sec-adv}. However, it is important
that the interrupt manager provide mutual exclusion between the interrupt
handler and the tasks with which it shares data.

Interrupt handlers are very special, because they are very closely
related to the hardware of the system.  An interrupt handler may not
be interrupted by a task, because the LWP system disallows preemption
(automatically) during the course of an interrupt.  Interrupt handlers may 
be interrupted by other interrupt handlers.  The priority scheme of
these interrupts is specified by the hardware (and is not relevent to or 
controllable by the LWP system).  
Interrupt handlers and tasks can prevent all {\em hardware} interrupts 
\index{hardware interrupts}
from occurring by {\em disabling}\index{disable}
interrupts to the CPU.  Interrupts
must be {\em enabled}\index{enable}
to continue normal processing.

\section{Basic Primitives}\index{primitives, LWP}\index{LWP primitives}
\label{sec-basic}
The states of tasks, interrupts, and the hardware in the LWP system can be 
controlled by a very small
set of primitives.  Higher level primitives will be introduced later,
but those primitives are merely built upon the fundamental synchronization
techniques described here and the fact that all tasks and interrupt handlers
share the same address space.

\subsection{Task State Control}
\label{sec-ctrl}
Creation and destruction are introduced here, although they are discussed
more thoroughly later.  All of these primitives may be executed only
by tasks unless otherwise specified.
\begin{description}
\item[\proc{Tasks.Create}]
    invokes a new task given a stack size, procedure to execute (for the
    control), and a priority.  The task proceeds along asynchronously from
    its creator and bears no further relationship to its caller; 
    all tasks are treated equally.  The caller receives a handle with which
    it can identify the newly created task.
\item[\proc{Tasks.Suspend}]
    puts the calling task in the suspend state.
\item[\proc{Tasks.Ready}]
    puts a task into a ready or running state under the rules of preemption
    described in section \ref{sec-sched}.  
    This procedure is passed a task and can be called by
    a task or by an interrupt handler.
\item[\proc{Tasks.SetPriority}]
    allows dynamic modification of a task's priority.  This technique can
    be used by an application specific scheduler (window manager, for 
    instance).
\end{description}

\subsection{Preemption Control}\index{preemption control}
These primitives must be used very sparingly, because they stop
the normal scheduling in the system.  In general, they are needed only
for building other inter-task communication primitives.  Both of 
these primitives can be nested.
\begin{description}
\item[\proc{DisallowPreemption}]
    turns off task switching.
\item[\proc{AllowPreemption}]
    turns on task switching.
    If the running task is of lower priority than a ready task, 
    the running task is preempted by the highest priority ready task.
\end{description}

\subsection{Interrupt Control}\index{interrupt control}
Interrupts can be controlled at several levels.  The Intel 8088 processor
can be put into a state in which it ignores interrupts.  The primary means
of mutual exclusion between tasks and interrupts is by enabling and
disabling the 8088's interrupt state.  The Intel 8259
interrupt processor has many levels of control.  The control of the 8259
is not discussed here, because it is very implementation specific and
effects only the relationship of hardware interrupts.  Tasks may
create interrupt handlers, but interrupt handlers cannot create other
interrupt handlers or tasks.
\begin{description}
\item[\proc{Interrupts.Create}]
    associates a importer-supplied procedure with an importer-supplied
    interrupt vector.  The import can also provide a pointer (object)
    to be passed to the procedure each time an interrupt occurs.  The
    procedure is called only if and when an interrupt occurs.  It
    returns a handle to be used by the following primitive.  Successive
    calls with the same interrupt vector displace the previous handler.
\item[\proc{Interrupts.Destroy}]
    performs the inverse action of \proc{Interrupts.Create}; it 
    dissociates
    the importer-supplied procedure from the importer-supplied interrupt.
    The caller need only supply the interrupt handler.  The previous
    handler is replaced in the interrupt vector to allow nesting.
\item[\proc{SYSTEM.DISABLE}]
    turns off interrupts to the 8088 (except for non-maskable
    and software interrupts).  No further interrupts will be detected
    until the following call is used.  This call does not nest.
    \footnote{
	It should be noted that {\em pushing} and {\em popping} the
	\index{interrupt flags}
	interrupt state is a cleaner solution to interrupt control, but
	the only modules in the Display which need to control interrupts
	are interrupt managers. Thus, this method should be sufficient.}
\item[\proc{SYSTEM.ENABLE}]
    turns on interrupts to the 8088.  This call does not nest.
\end{description}

\section{Advanced Primitives}\index{primitives, LWP}\index{LWP primitives}
\label{sec-adv}
The discussion thus far has described a set of primitives which allow
very simple communication between tasks and interrupts. 
This section discusses the primitives normally used by tasks:
the so-called ``higher level'' facilities.  These primitives allow
the programmer to better model the real-time environment, because they
provide a cleaner method of communication between tasks.

\subsection{Time}
\index{TaskTime}
\index{Time}
Tasks can suspend themselves for a finite period of time.
The timing is not exact, but is approximately the time
specified by the caller.  The timed suspensions
are identical to regular suspensions in semantics; 
a task may be readied (``awoken'') at any point.

\subsection{Messages}
\index{TaskMessages}
A \newterm{Message} is a single two-way synchronized data pass
between a sending task and a receiving task.  
The \newterm{sender} suspends itself while the receiving task processes 
the message.
When the \newterm{receiver} replies to the message, the sender is awoken
by the receiver and both tasks continue asynchronously.
When there are no messages awaiting reception
for a particular receiver, the receiver suspends itself until it
is awoken by a sender.  Messages are not ordered, but are serviced
on a first-come-first-serve basis.

There are two parts to a message: a \newterm{request} and a \newterm{reply}.  
Both 
parts may be of arbitrary size and are treated as opaque entities by the
message manager.  The sending task must provide the send buffer and
the reply buffer at the point it enters the message manager.  

The receiver
need only supply a request buffer to await a message.  The size of the
request buffer must be big enough to hold the sender's buffer.  It is
the receiver's responsibility to define the message sizes. Once the
receiver is finished processing the request, it replies to the sender
by supplying a buffer to the message manager.  At this point, the
sender is readied and may preempt the receiver, if the sender is at a higher 
priority.

This system also provides a mechanism for {\em timed} message receptions.
The receiver can specify a maximum time period in which any message
can be received or the attempted reception will return with failure.  
Message
sends may not be timed, because it is not as simple to unblock the sender
while the receiver is processing the reply.  

\subsection{Monitors}
A \newterm{Monitor} provides a method of excluding all but one task from a
critical section of code.
Protection is implemented by
surrounding the critical section with \newterm{enter} and 
\newterm{exit} calls.  
When multiple tasks attempt to enter a critical section controlled by 
a monitor, the first is allowed to enter and the others are 
{\em queued} and suspended at the entry on a first-in-first-out basis.
When a task exits a monitor, it dequeues and readies the first task 
awaiting entry (if there is one waiting).  
Monitor entries may be timed as message
receptions are and a timeout is indicated by failure to enter the critical
section.  Exits do not suspend the caller; thus, they need not be timed.

\section{Creation and Destruction}
\index{Creation}
\index{Destruction}
Interrupt handlers are 
easy to create and destroy, because they execute only for brief periods of
time and a task never executes during an interrupt.  Therefore, a task 
can safely create and destroy interrupt handlers without the complication
of disrupting a control thread which it is destroying.
Interrupt handlers cannot create or destroy tasks or other interrupt handlers.

Creating tasks is quite simple, but destroying them is a little more
complicated.  Tasks can be created only by other tasks. 
The creating task may be preempted by the task it creates, 
if the new task is of higher priority.

The only way to destroy a task is for the task to destroy itself.
This system is not a general purpose time sharing environment and/or
command execution environment.  It has been designed for 
the purpose of real-time applications.   Much complexity is added when
a task can be terminated at any point during its lifetime.\footnote{
    The task could be in a critical section or have allocated resources
    which only it can free.  Maintaining proper system state after such
    a termination in the fragile environment of the IBM PC and MS-DOS is 
    difficult if not impossible.}
If a task fails a run-time assertion test (e.g. a range check), 
it will appear as if the task had committed suicide.

Tasks which commit suicide (via \proc{ProgErr.Terminate}) 
are treated as abnormal deaths and tasks which \code{RETURN} are 
considered to have terminated normally.  
This distinction is important, because a task which terminates
abnormally will cause termination of the entire system.

To summarize, a simple step by step enumeration of a task's life follows:
\begin{enumerate}
\item
    A task calls \code{create} with a procedure, priority, name, and stack size.
\item
    After the task is readied, the new task is made known to task information
    groups.  The notification is made from the newly created task before
    it enters its normal execution thread.  During this notification, the
    new task's creator is suspended.
\item
    Upon return of the notification, the new task's creator is readied
    by the new task and they proceed along asynchronous paths.
\item
    The task proceeds until it terminates by:
    \begin{enumerate}
    \item \label{en-normal}
	executing a \code{RETURN} from the procedure which was passed into
	the create procedure, or
    \item \label{en-abnormal}
	encountering a run-time error or committing suicide.
    \end{enumerate}
\item
    If the task terminated normally, the task notifies the task
    information groups that it is dying.  The system terminates when 
    the last task terminates.
\item
    If the task terminated abnormally (case \ref{en-abnormal}),
    then the system will terminate.  Normal termination (case \ref{en-normal})
    causes no global modification of the system state, except that there is
    one less task running.
\end{enumerate}

\section{Per Task Information}
All control threads in this system run in a shared memory space.
Clearly, tasks need their own ``private'' memory areas so that they
can execute normally (i.e. stack and state information).
Modula-2 provides information hiding on a per module basis.
Information hidden in modules sometimes needs to be managed on a 
per task basis.
Instead of providing specialized entry points for a specific set
of priviledged modules, a generalized task information facility is
provided.  

The method is to provide a two dimensional table indexed by task identifier
and a unique key representing a class of data.  A module which manages
a resource can allocate a key during its initialization and then use
that key to access its private data area for any task in the system.
Specifically, each task will have an array of pointers.  The pointers
are supplied by the data class managers and the keys (indexes into the
array) are managed by the LWP system.  
To access a specific entry, a task identifier and
a key is supplied and the data manager receives the locations value or
may modify it.

\section{Modular Decomposition}
This component is decomposed as follows:
\begin{description}
\item[\module{Tasks}]
    The primary module of concern to the typical programmer.  It contains
    the primitives for creation and the fundamental processor sharing 
    primitives (Ready, Suspend).  Provides low-level procedures for preemption
    control.
\item[\module{TaskMessages}]
    Provides the synchronous data passing primitives.
\item[\module{TaskMonitors}]
    Provides the mutual exclusion primitives.
\item[\module{TaskTime}]
    Supports ``timed'' suspends of tasks.
    Provides a method of mutual exclusion for code which runs in DOS
    which it registers with the \module{DOS} module.
\item[\module{TaskInfo}]
    Used by modules which need access to information on a per task basis.
    Exports methods for creating and destroying data classes and for accessing
    the data values for a particular task or the current task.  
\item[\module{Interrupts}]
    Provides access to the hardware and software interrupts on the IBM
    PC.  Contains primitives for creation and destruction of interrupt
    control threads.
\item[\module{TasksPRIVATE}]
    This module is imported only by \module{TaskInfo} and \module{Tasks}.  
    It {\em is} the guts of the LWP environment.  It exports stack 
    formats and the actual data structures associated with a task 
    as well as the variables to manage tasks.
\end{description}

\section{Run Time Support/Integration}

The current implementation of the Logitech RTS will be
modified to support the requirements of the tasking system.  
The tasking system supports its own version of process contexts,
because it needs to integrate with \module{Interrupts}.
This model may not work with the RTD, PMD, or RTS.  A 
new \module{Storage} module is required, because Logitech implements
a heap storage model which is integrated with its process model.

\section{Debugging}\index{debugging, task}

To expedite debugging, there is a way to dump the state
of a task (including a stack trace) to the display or a file.  
There will be hooks for providing task switch tracing information to
simplify debugging, but the actual facility will only be implemented
as required.  The debugging facilities are primitive and can be improved
as needed.

\section{Configuration}

\subsection{Hardwired Configuration}\index{hardwired configuration}
\index{configuration, hardwired}
The following parameters will be compiled into this component:
\begin{itemize}
\item
    Maximum and minimum priorities.
\item
    Maximum, minimum and default stack sizes.
\item
    The names, priorities, and stack sizes of the tasks internal to the system: 
    idle task and death notifier.
\item
    The stack size and vector of the timer interrupt handler.
\item
    Maximum number of TaskInfo keys.
\item
    The resolution and range of the timeout facility.
\item
    Maximum size of task name.
\end{itemize}

\subsection{Init-time Configuration}\index{init-time configuration}
\index{configuration, init-time}
The question of initialization is an interesting one, because the tasking
system does not need any other information except as is provided above.
Importers will need access to an organized list of parameters, 
because a task's parameters are not independent of the rest of the system.  
Specifically, a task's priority affects how it will interact with other
tasks and its stack size determines how much shared memory is included
for its private memory space.  To ensure an organized environment,
the list of information shared by importers is:
\begin{itemize}
\item
    task names.
\item
    Task priorities (and modification values).
\item
    Stack sizes.
\item
    Interrupt vectors (some of these may be hardwired by the importers).
\end{itemize}

Because the names of the tasks can be generated by the importers, the
tasking system can use these as keys to gain access to the other parameters
which are under this component's domain.  There is a new module called
\module{TaskParameters}\footnote{Not supplied with the library.} which will 
provide an entry point to look 
up task parameters
by name.  The reason that this is a separate module is that this really
has nothing to do with the LWP component per se.  The format of the
configuration can be established from the definition modules of the rest
of this component and thus \module{TaskParameters} can certainly be an 
independent entity.

\subsection{Run-time Configuration}\index{run-time configuration}
\index{configuration, run-time}
This component does not need to be dynamically reconfigured.

\section{Errors}
All of the errors that can occur in this component are assertion faults.
Because the IBM PC is a fragile entity, corrupted memory is a common problem.
This component assumes that exporters which pass in invalid parameters
could not correct a failed call, thus it seems fair to treat all illegal
calls as  assertion faults.

\subsection{Exporter Errors}\index{exporter errors}\index{errors, exporter}
The following is a list of externally generated errors:
\begin{itemize}
\item
    Passing in an invalid object (be it a task, interrupt, or monitor).
\item
    Attempting to destroy a monitor on which tasks are blocked.
\item
    Attempting to create a task with invalid parameters (stacks too large, 
    illegal priority, etc.)
\item
    Calling \proc{Suspend} in a non-preemptable section.
\item
    Executing any primitive other than \proc{Ready} from an interrupt 
    thread.
\item
    Too many \module{TaskInfo} keys requested.
\item
    Stack overflow, range errors, and the like.
\end{itemize}

\subsection{Importer Errors}\index{importer errors}\index{errors, importer}
The following is a list of internally generated errors:
\begin{itemize}
\item
    Various internal data structure corruptions.
\item
    Task creation/destruction Notices are invalid.
\item
    Out of memory.
\end{itemize}

\section{Testing}
The design of this system was made flexible explicitly to increase testability.
The testing can be broken down easily:
\begin{enumerate}
\item
    The assumptions of Logitech's Run Time System need to be verified
    (preliminary work has already begun).
\item
    The high speed process model needs to be tested to establish if the
    context switching code is valid.  The following tests should use both
    models of processes.
\item
    Interrupts use the same context switching code (with minor modifications)
    as the high speed process model and it really does not need any ``tasks''
    to validate most of its correctness.
\item
    Task creation and destruction.
\item
    Fundamental primitives (Ready and Suspend). 
\item
    Preemption control primitives.
\item
    TaskInfo.
\item
    TaskTime.
\item
    Monitors.
\item
    Messages.
\item
    The last test should consist of a complex interaction of all of 
    the primitives.  
\end{enumerate}


