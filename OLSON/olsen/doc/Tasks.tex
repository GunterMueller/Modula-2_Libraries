% Copyright 1988 by Olsen & Associates (O&A), Zurich, Switzerland.
%
%		    All Rights Reserved
%
%
% Permission to use, copy, modify, and distribute this software and its
% documentation for any purpose and without fee is hereby granted,
% provided that the above copyright notice appear in all copies, and
% that both that copyright notice and this permission notice appear in
% supporting documentation, and that all modifications of this software
% or its documentation not made by O&A or its agents are accompanied
% by a prominent notice stating who made the modifications and the date
% of the modifications.
%
% O&A DISCLAIMS ALL WARRANTIES WITH REGARD TO THIS SOFTWARE AND ITS
% DOCUMENTATION, INCLUDING ALL IMPLIED WARRANTIES OF MERCHANTABILITY AND
% FITNESS.  IN NO EVENT SHALL O&A BE LIABLE FOR ANY SPECIAL, INDIRECT OR
% CONSEQUENTIAL DAMAGES, ANY DAMAGES WHATSOEVER RESULTING FROM LOSS OF
% USE, DATA OR PROFITS, WHETHER IN AN ACTION OF CONTRACT, NEGLIGENCE OR
% OTHER TORTIOUS ACTION, ARISING OUT OF OR IN CONNECTION WITH THE USE OR
% PERFORMANCE OF THIS SOFTWARE OR ITS DOCUMENTATION.
%%%%%%%%%%%%%

\chapter{Lightweight Processes}
\label{lwpchapter}

\section{Purpose}

This chapter explains lightweight processes and gives some advice
on their use.  The \newterm{lightweight process environment} sets up
multiple processes, called \newterm{tasks}, which use the same address
space and hence interact as concurrent processes.

Don't expect a tutorial on all aspects of concurrent programming, however.
That's too big.  We suggest you see the bibliography for some key references 
instead.

\section{Lightweight Process Environment Structure}
\label{structure}

Let's look at lightweight processes in more detail.  We begin with the 
terminology.

\subsection{Threads, Events, Tasks and Interrupt Handlers}

A \newterm{thread} is a CPU-driven stream of execution which runs without
external interruption by hardware and follows the execution path dictated by
a deterministic code body.  We use the term ``thread'' to include both
tasks and interrupt handlers.

Every thread has a \newterm{stack} which holds local variables and local
state for the thread.  When the CPU executes a thread, its stack pointer(s)
point to the thread's stack and its instruction pointer points to the next
instruction which the thread should execute.

An \newterm{event} is an action which, when it happens, may change which
thread the CPU will execute next.  Lightweight processes manage events.
We divide events into two types: external and internal events.  

An \newterm{external event} occurs at an arbitrary moment which no thread 
can predict.  For example, hardware generates an interrupt when it detects 
an external event.
We include ticks of the system clock as ``external events'', even though
these events occur at regular intervals.

An \newterm{internal event} is an event that is triggered by a thread.  Since
a thread triggers an internal event, it can use the knowledge that it will
cause the event by preparing states and conditions before the event occurs.
An internal event amounts to a \newterm{synchronization point} between
threads that the thread initiating the event sets up and manages.
An important example of an internal event is the return from an interrupt
handler that has readied a task to take over the CPU.

\newterm{Interrupt handlers} are threads that react to external events.
Each handler is attached to a \newterm{vectored interrupt} on the local
machine; hardware triggers these interrupts.

Although interrupt handlers can trigger internal events, \newterm{tasks}
react to and manage internal events.  Each task is a thread and has a
\newterm{priority} that determines which tasks it can \newterm{preempt},
i.e., take over the CPU from.  Priority constitutes a right acknowledged
by all tasks and interrupt handlers that, when an event occurs, if it wants
to run, the highest priority task which has waited longest to run should get
the CPU.

Interrupt handlers have higher priorities than tasks, since any interrupt
transfers the CPU to an interrupt handler, no matter which task is
executing.  Interrupts have priorities in relation to each other, specified
by system hardware, so one interrupt handler can preempt another.

\subsection{Designing with Interrupt Handlers and Tasks}

Why are tasks sometimes preferable to interrupt handlers?  Because you 
can create as
many tasks and internal events as you need, you can plan and prepare for
internal events, and tasks don't react to external events.

Why are interrupt handlers sometimes preferable to tasks?  Because they 
react immediately
to received external data, even if it comes in fast.

Good team program designs use interrupt handlers to get external data fast.
The interrupt handlers then trigger internal events, which wake up tasks
to read and interpret the data.  The tasking design uses internal events
and relative priorities between tasks to assign which work gets done first,
and which task(s) should do it.  The lightweight process environment manages
the necessary preemption to get fast transfers of the CPU between tasks.

Because task creation and termination are both cheap, your design should 
dynamically
create as many tasks as it needs to keep its problem management simple and
under control.

\subsubsection{Tasks in Detail}

The module \module{Tasks} manages tasks.  Each task is an instance of the
object \code{Tasks.\-Object}.  Calling \proc{Tasks.\-Create} creates a task
with a specific \code{priority}.  You should also tell \proc{Tasks.\-Create}
how large you want the stack of the task to be.

The lightweight process environment is initialized by the main program
thread, which is composed from the initialization bodies of all the modules
\code{IMPORT}ed by the program.  This main program thread becomes its own
task, and gets \code{TaskConsts.\-default\-Priority} as its priority.

A task is always in one of three states:
\begin{itemize}
\item
 \newterm{Running}.  The task which the CPU is currently executing is
 running.  Only one task in the lightweight process environment
 can be in this state.
\item
 \newterm{Ready}.  When a task is ready, it can become the running task
 at a synchronization point, if at this moment it is the task in the set
 of ready tasks which has the highest priority. If more than one task
 has the highest priority, it is the task which has been ready for the
 longest period.
\item
 \newterm{Suspended}.  A suspended task is not eligible to run.  Tasks
 and interrupt handlers can \newterm{ready} a
 suspended task by calling the procedure \proc{Tasks.\-Ready}.
\end{itemize}

The procedure calls \proc{Tasks.\-Suspend} and \proc{Tasks.\-Ready} reset the
state of a task.  \proc{Tasks.\-Suspend} changes the state of the running
task from running to suspended.  \proc{Tasks.\-Ready} takes a single
\code{task} argument and changes its state to ready.  If its state
is already ready, then \proc{Tasks.\-Ready} achieves nothing.

When either \proc{Tasks.\-Suspend} or \proc{Tasks.\-Ready} is called, if
preemption is enabled (see section \ref{overriding}), a synchronization
point is reached, and a decision is made as to which task should become the
running task.  If preemption is disabled, a synchronization point is readied
only if \proc{Tasks.\-Suspend} is allowed.  In either case, the new running
task is the eligible task with the highest priority.  If there are no
eligible tasks, the lightweight processes system runs the idle tasks.
The idle task is a task internal to the LWP --- its sole function is to
provide ``something to run'' if all other tasks have been suspended.

If a new task should become the running task, a \newterm{task switch} occurs:
CPU and old task state is saved, new task and CPU state is restored, and the
new task becomes the running task.

\subsubsection{Interrupt Handlers in Detail}

Interrupt handlers are not tasks but Modula-2 procedures that are called
each time a \newterm{vectored interrupt} occurs.  System hardware triggers
the interrupt, and code from the \module{Interrupts} module then calls the
relevant interrupt handler.

An interrupt handler handles the conditions associated with the external
event which triggers it.  For instance, if the external event is the arrival
of a character at a UART port, the interrupt handler reads the
character.

After handling the conditions associated with the external event, the
interrupt handler generates an internal event to inform a task that
the external event has occurred.  To keep things very safe and simple, the
lightweight process environment makes only {\em one\/} call available to
interrupt handlers so that a handler can inform a task: \proc{Tasks.\-Ready}.

An \newterm{interrupt manager} is usually a module that contains the
interrupt handler and provides entry points for tasks to access the data
associated with the interrupt.  Tasks call the interrupt manager and are
either passed the data or suspended until the data is received {\em via\/}
an interrupt and the interrupt handler then calls \proc{Tasks.\-Ready}.

How this sharing mechanism is implemented is not important to this discussion,
but it is
important that the interrupt manager provides mutual exclusion between
the interrupt handler and the tasks with which it shares data.  The interrupt
manager can support mutual exclusion by memory-based means or by adjusting
priorities, as described in the next subsection.

The module \module{Interrupts} manages interrupt handlers.  It offers calls
to install or de-install an interrupt handler.  The de-installation call
(\proc{Interrupts.\-Destroy}) restores the previous handler for that interrupt.

Interrupt handlers always run with preemption disabled, so that no task can
preempt an interrupt handler.  However, one interrupt handler can preempt 
another
interrupt handler.  The only \module{Tasks} call that an interrupt handler
can call is \proc{Tasks.\-Ready}, so no interrupt handler can call any higher
level concurrency primitive which uses calls in \module{Tasks} other than
\proc{Tasks.\-Ready}.  If these prohibitions are violated, the \module{Tasks}
module will cause abnormal termination.

\subsection{Adapting Priorities}

The priorities available to threads form a total ordering, from the lowest
task priority to the highest task priority, followed by the lowest interrupt
priority to the highest interrupt priority.

Occasionally, a thread needs to change its priority so that it can exclude
other threads from the right to preempt its work.  The lightweight process
environment provides calls to revise priorities, and the local machine or
language also provides means.

\subsubsection{Temporary Overrides}
\label{overriding}

We describe here means by which
\begin{itemize}
\item
 a running task can become the highest priority task, or
\item
 an executing thread (running task or interrupt handler) can become the
 highest priority thread.
\end{itemize}

Each of these means are {\em temporary\/}, since each thread which shifts its
priority can always return to its old level.  Each means is also connected
only with the running thread.  If the running thread suspends itself, its
priority shift does {\em not\/} carry over to the next running thread and
its priority shift is {\em re-established\/} when the thread again
becomes the running thread.

A running task can make itself the highest priority task by
\newterm{disabling preemption}.  Only interrupt handlers can preempt a task
which disables preemption.  If a task disables preemption, it must
\newterm{enable preemption} at some later time, or no other task will get a
chance to run.

Call \proc{Tasks.\-Disallow\-Preemption} to disable preemption.  Call
\proc{Tasks.\-Allow\-Preemption} to enable preemption.  These calls are
\newterm{nested}: \proc{Tasks.\-Disallow\-Preemption} increments a nesting
count; \proc{Tasks.\-Enable\-Preemption} decrements it.  Only when the
nesting count goes down to zero does \proc{Tasks.\-Enable\-Preemption} really
enable preemption.  \proc{Tasks.\-Enable\-Preemption} generates a runtime
range checking error if it is called while preemption is enabled.

The main use of disabling preemption is to deal with tasks and to implement
higher-level inter-task communications primitives.  If you get some state
from a task and want to then use this state, you must disable preemption
before you get the state and keep preemption disabled until you interpret
the state.  If you don't, this state could change between the time you get
it and the time you interpret it.

Interrupt handlers always run with preemption disabled.  A synchronization
point occurs whenever an interrupt handler returns and the running task has
preemption enabled, or whenever a task enables preemption.  At this point,
a new decision of which task should run is made, and a task switch may
occur.  If the task switch does occur, then the new running task
\newterm{preempts} the old running task.

A running task, or an executing interrupt handler, can make itself the highest
priority thread by \newterm{disabling interrupts}.  When interrupts have been
disabled, no interrupt will trigger a new interrupt handler until interrupts
are again enabled.

On the IBM-PC, Logitech Modula-2 system, calling \proc{SYSTEM.\-DISABLE} will
disable interrupts, and calling \proc{SYSTEM.\-ENABLE} will enable them again.
These calls are {\em not\/} nested, so if you call either 
\proc{SYSTEM.\-DISABLE}
or \proc{SYSTEM.\-ENABLE} twice in a row, the second
call achieves nothing at all. 

Because the calls to disable and enable interrupts are {\em not\/} nested,
any procedure call you make in code surrounded by \proc{SYSTEM.\-DISABLE} and
\proc{SYSTEM.\-ENABLE} statements may access code that enables interrupts again.
This surprise can happen, for instance, when you call \code{DOS}.

\subsubsection{Permanent Resets}

You can reset the priority of a task --- any task, not just the running
task --- by calling \proc{Tasks.\-Set\-Priority} with a \code{task} argument.
To find out the priority of a task, call \proc{Tasks.\-Get\-Priority}.  To
get the \code{Tasks.\-Object} that is currently running, call
\proc{Tasks.\-Get\-Current}.

To set interrupt priorities, many operating systems offer calls and
assembly language interfaces for \newterm{masking interrupts}.  A masked
interrupt is disabled until it is unmasked.

Often a system will let you associate an \newterm{interrupt mask} with each
vectored interrupt.

The interrupt mask defines a set of masked interrupts, which the system
immediately masks upon the occurrence of an interrupt.  By setting interrupt
masks appropriately, you can rearrange your interrupt priorities according
to any scheme you like.

An interrupt mask applies only to a single thread at a time --- either the
running task or the interrupt to which it is directed.  The mask will change
when the thread changes and will be restored when the thread executes again.

\subsection{When Events Happen}

A list of the times when one thread changes to another might help you in
thinking about your program's mechanics:
\begin{itemize}
\item
 When an interrupt occurs, an interrupt handler executes.  This handler can
 call \proc{Tasks.\-Ready} to ready other tasks.
\item
 When an interrupt handler returns, a task resumes control.  If the running
 task when the interrupt occurred has disabled preemption, it will run again.
 
 Otherwise, the standard rule comes into effect: the highest priority, longest
 waiting to run task gets the CPU.  Perhaps this task is the running task --- 
 it
 definitely is if the interrupt handler never called \proc{Tasks.\-Ready}.
 If it did call \proc{Tasks.\-Ready}, readying a task whose priority exceeds
 the priority of the running task, then this new task will run.
\item
 When a task calls \proc{Tasks.\-Ready} and preemption is enabled, a task
 switch occurs if the \code{task} argument has a priority which is
 higher than the priority of the running task.
\item
 When preemption has been disabled, and is enabled again by a call to
 \proc{Tasks.\-Allow\-Preemption}, a task switch may occur.
\item
 If a task calls \proc{Tasks.\-Suspend}, another task will take its place as
 the running task.  If no tasks are ready, the environment will abnormally
 terminate.
\item
 If a task returns from its main procedure, it dies, and another task takes
 its place.
\end{itemize}

\section{PC-DOS Complications}

Lightweight process environments:
\begin{itemize}
\item
 \newterm{Fit inside} another operating system.  A lightweight process
 environment uses calls offered by a surrounding operating system, instead
 of building its own calls to replace operating system functions.
\item
 \newterm{Extend} the operating system inside of which they fit.  A program
 running in the lightweight process system works inside a virtual machine,
 defined by operating system calls {\em plus\/} calls implemented by the
 lightweight process environment.
\item
 Are \newterm{fragile}, in that they can't stop a programmer from using (or
 abusing) the surrounding operating system to make the lightweight process
 environment blow up.
 
 There are always \newterm{conventions}, which the system can't enforce, but
 which a programmer must follow nonetheless, in order that programs in the
 lightweight process environment can run without blowing up.
\end{itemize}

The PC-DOS operating system surrounds the lightweight process
environment, which runs on IBM-PC machines and clones.  PC-DOS conventions
modify the model of lightweight processes which we've so far introduced.

\subsection{Calling \code{DOS} --- the Non-Reentrancy Problem}

In the Modula-2 Logitech system, if you want to access the operating system,
you will ultimately call \proc{SYSTEM.\-DOSCALL}.  This procedure calls
\code{DOS} and leaves return information in specific registers, as
described in IBM-PC manuals.

The call to \code{DOS} does not enter any special ``kernel'' level of code
--- it operates entirely at the priority level of the task or interrupt
handler which calls \code{DOS}.  So another task could preempt your task
while it executes a call to \code{DOS}.

Fine, you might think.  But a problem arises if the task which preempts
you, or an interrupt handler, calls \code{DOS} while your task or interrupt
handler is already executing a call to \code{DOS}.  Your system may crash.

There are code sections within \code{DOS} code, called \newterm{critical
sections}, in which, if you call \code{DOS} while another procedure is 
already executing code from one of these sections, \code{DOS} will cause 
abnormal termination.  \code{DOS} is not
\newterm{re-entrant}; it won't always survive code execution by more than
one call at a time.

\subsection{The \code{DOS} Monitor}

We developed a lightweight process environment module \module{DOSMonitor} to
deal with the non-reentrancy of \code{DOS}.  Before you make one or more
calls to \code{DOS}, call \proc{DOSMonitor.\-Enter}.  After these calls have
finished, call \proc{DOSMonitor.\-Exit}.  Put calls to \code{DOS}, i.e.,
\proc{SYSTEM.\-DOSCALL}s in Logitech Modula-2, only between ``enter'' and 
``exit'' calls.

If you call \code{DOS} between these ``enter'' and ``exit'' calls, 
it will not get called while some other procedure is within a
critical section (an undocumented, but well-known ``critical section'' flag
in \code{DOS} helps us out here $\ldots$).

\section{Concurrency Primitives}

We've given the low-level story, and we've explained some of the trickiness
of working within \code{DOS}.  We still have to tell you the higher level
story.

Since the lightweight process environment implements concurrent tasks,
it's important that it also implement the basic primitives which any
standard book on concurrent programming or real time systems will tell
you are essential to getting such a system running.

At this point, we have to assume you know something about concurrent
programming, because we may start using terms which you probably won't
understand without an introduction to the subject.  

\subsection{Atomic Actions with Monitors}

In keeping with the library's policy to not change Modula-2 as a 
language, the lightweight process environment has implemented
/em{monitors} as procedure calls \proc{TaskMonitors.\-Enter}
and \proc{TaskMonitors.\-Exit}.  Call \proc{TaskMonitors.\-Create}
to create a monitor (\proc{TaskMonitor.\-Object}.

A monitor surrounds a section of code, as the following template demonstrates:
\begin{verbatim}
    VAR
        monitor: TaskMonitor.Object;

        ...
    PROCEDURE XXX;
        BEGIN
            TaskMonitors.Enter( monitor );
    
            ... code ...
    
            TaskMonitors.Exit( monitor );
        END XXX;

    BEGIN
        TaskMonitors.Create( monitor );
    END Module.
\end{verbatim}

A monitor enforces the restriction that only one task may execute a code
section at a time.  If a task is executing a code section and another task
tries to enter, the monitor will make the second task wait by suspending it.  
If further tasks
try to enter, they must also wait.  Tasks enter the monitor (i.e., start to
execute code surrounded by the monitor) in first to wait, first to enter
order.

A useful way to look at these \newterm{critical sections} of code --- code
which would create inconsistent state if more than one task executed it at
a time --- is as \newterm{atomic actions}: steps you have designed to execute
in one burst, independent of other ongoing steps in your program.  Your
higher-level design can see these steps as single indivisible instructions.

Module \module{Task\-Monitors} handles monitors.  In addition to standard
\proc{Enter} and \proc{Exit} calls, \module{Task\-Monitors} also has a
\proc{TimedEnter} call, which gives up if it waits longer than
\code{milliseconds} milliseconds.  \proc{Timed\-Enter} returns a 
\code{BOOLEAN},
to inform the caller if it succeeded or gave up.

\proc{Enter} and \proc{Exit} are nested calls, so you must follow multiple
enters with multiple exits before the next task can enter the code section
which a monitor surrounds.  If \proc{Exit} is called without a corresponding
\proc{Enter} having been called (or a successful \proc{Timed\-Enter}) for the
same \code{TaskMonitor.\-Object}, it will trigger abnormal termination.

\subsection{Timeouts --- Putting Tasks to Sleep}

It's better to put your task to sleep than to continuously poll
until some condition becomes \code{TRUE}.  Then other tasks can use the
CPU too.

Module \module{TaskTime} lets you suspend your task for a specific period
of time.  The call \proc{Sleep} suspends the current task, telling an
interrupt handler associated with the system clock to ready the task again
once a certain number of clock ticks have passed by.

If you want to wake up a sleeping task early, call \proc{Wakeup}.

The \proc{Sleep} call takes as a parameter how long you want the running
task to sleep, in milliseconds.  \proc{Sleep} then translates these
milliseconds into system clock ticks, by truncating: the milliseconds you
request are divided by the number of milliseconds {\em per\/} clock tick,
and any remainder is thrown out.

If you call \proc{Sleep} with a 0 milliseconds argument, we assume you want
to sleep a little, i.e., give someone else a chance to run while you poll,
so \proc{Sleep} will put the running task to sleep for one clock tick.

To find out how many milliseconds your system has {\em per\/} clock tick,
call \proc{Resolution}.  It returns the number of milliseconds as a
\code{REAL}.

\subsection{Messages and Synchronization}

The lightweight process environment offers the \newterm{rendezvous} concept
which the Ada programming language adopted some years ago.

The module \module{TaskMessages} has three basic calls: \proc{Send},
\proc{Receive}, and \proc{Reply}.  Message passing is \newterm{synchronous}:
no task can send a second message unless it has received a reply to its first
message.  The advantages of synchronous message passing are that it is easy
to implement with procedures and that the sending side has no buffer storage
or acknowledgement problems.

The lightweight process environment implements synchronous message passing
by suspending the sending task.  The sending task calls \proc{Send},
which contains three arguments:
\begin{itemize}
\item
 the task which will receive the message, called the \code{recipient}
  (\module{Task\-Messages} guarantees delivery);
\item
 the message, called the \code{request}, contained in an \code{ARRAY OF
 SYSTEM.\-BYTE}; and
\item
 a buffer for the reply, called \code{reply}, another \code{ARRAY OF
 SYSTEM.\-BYTE}, which must be large enough to receive any reply the
 recipient might choose to make.
\end{itemize}

\proc{Send} only returns to the sending task once the message has been
received, and once the task which receives the message has replied to
the message.  Until that time, the sending task cannot progress to the
instruction which follows the call to \proc{Send}.

\proc{Send} puts its message on a \newterm{message queue} owned by the
recipient task.  This task calls \proc{Receive} to get a message from
its message queue.  It receives both
\begin{itemize}
\item
 the task which sent the message, called the \code{sender}; and
\item
 the message, called the \code{request}, which \code{sender} sent
 by calling \proc{Send}.
\end{itemize}

When it receives a message, the recipient task knows that task \code{sender}
has blocked and is waiting for a reply from the recipient task and no other.
The recipient task has complete control over when \code{sender} will become
ready to run again, and knows this when it receives the message.

A task can call \proc{Receive} as often as it wants, without having to
reply to any of the messages it receives.  The more calls to \proc{Receive}
it makes, the more it will know about specific tasks which have blocked
and are waiting for replies.

The task which calls \proc{Receive} {\em also\/} blocks, but only if its
message queue is empty.  It blocks until another task sends it a message
by calling \proc{Send}.  Because we don't always want tasks to block for
what might be forever, \module{Task\-Messages} provides a \proc{Timed\-Receive}
call to substitute for the \proc{Receive} call.

\proc{Timed\-Receive} quits waiting for a message after a specified number of
milliseconds, which you pass in the \code{milliseconds} parameter.  It
returns \code{TRUE} if it received a message, and \code{FALSE} if not.

When a task that has received a message wants to reply to the message,
it should call \proc{Reply}, passing to it
\begin{itemize}
\item
 the task to which the reply should go, called the \code{sender}, since
 it sent a message earlier; and
\item
 the reply message, called \code{reply}.
\end{itemize}

The sender, when it calls \proc{Send}, should know the maximum size of
the reply message.  If the reply message is too large,
\proc{Reply} will cause abnormal termination.  \proc{Reply} will also
cause abnormal termination if \code{sender} is not waiting for a reply
from the running task which calls \proc{Reply}.

Calling \proc{Reply} will make the \code{sender} argument ready to run
again.

\subsection{The Asymmetry of Rendezvous}

The interaction between \proc{Send}, \proc{Receive}, and \proc{Reply}
is known as a \newterm{rendezvous}.  It guarantees that all tasks that
send to another task will block before they get their replies, so it
offers to the recipient task an opportunity to \newterm{synchronize}
its data with the data of the tasks which send messages.

Synchronization is the main way in which messages between tasks get
used, since the shared memory makes information passing more or less
simple without the need for message mechanisms.

See the definition module of \module{TaskMessages} for more
detail.

The asymmetry of rendezvous works as follows:
\begin{itemize}
\item
 A sending task knows the name of the receiving task's \code{Tasks.\-Object},
 but the receiving task does not know the sending task's name until the 
message is received.
 
 The receiving task acts like a server:  it does not need to know the
 name of a client until the client requests a service.
\item
 A receiving task can receive more than one message before it makes any
 replies, but a sending task can only send one message, after which it
 must wait until the reply arrives.
 
 The receiving task has freedom of action and freedom to choose when and
 in what order it will reply to messages.  The sending task has no freedom
 at all.
\end{itemize}

\subsection{Task Specific Notices: Birth, Death, Environment Termination}

The module \module{Task\-Notices} exports procedures which export notices.  See
the definition module for further information.

Essentially, notices provide a service whereby modules that have an interest
in an event can \newterm{register} their interest, and modules that detect
events can \newterm{notify} these modules, to tell them that the event has
occurred.  The module that detects the event exports a procedure which
returns a \newterm{notice}, a \code{Notices.\-Object}.

Modules with an interest in the event call \proc{Notices.\-Register} to register
with the notice.  When it detects that the event has occurred, the detecting
module ``calls'' the notice, by calling \proc{Notices.\-Call} with the notice
as a parameter.

\module{Task\-Notices} exports three parameters:
\begin{itemize}
\item
 a \newterm{task birth notice}, returned by \proc{Get\-Birth}.  Procedures
 registered with this notice are called whenever a task is created.
 The \code{SYSTEM.\-ADDRESS} parameter to the procedure contains the
 \code{Tasks.Object} of the newly created task.
\item
 a \newterm{task death notice}, returned by \proc{Get\-Death}.  Procedures
 registered with this notice are called whenever a task dies (by
 returning from its main procedure).
\item
 a \newterm{environment termination notice}, returned by
 \proc{Get\-Exit\-Request}.  The procedure \proc{Was\-Exit\-Requested} returns
 \code{TRUE} if the environment termination notice was called, else
 \code{FALSE}.
 
 The idea behind this notice is that every task should die if the system is 
 to die
 normally.  Many tasks won't die until they notice that a task has requested
 termination.  Some tasks won't die until they become running tasks which run
 long enough to {\em try\/} to notice.
 
 The environment termination notice, if called, will cause future calls
 to \proc{Was\-Exit\-Requested} to return \code{TRUE}.  You should register
 procedures with this notice to guarantee that any tasks which are
 sleeping will be made ready again, so they can check if termination
 was requested and then die.
\end{itemize}

