% Copyright 1988 by Olsen & Associates (O&A), Zurich, Switzerland.
%
%                   All Rights Reserved
%
%
% Permission to use, copy, modify, and distribute this software and its
% documentation for any purpose and without fee is hereby granted,
% provided that the above copyright notice appear in all copies, and
% that both that copyright notice and this permission notice appear in
% supporting documentation, and that all modifications of this software
% or its documentation not made by O&A or its agents are accompanied
% by a prominent notice stating who made the modifications and the date
% of the modifications.
%
% O&A DISCLAIMS ALL WARRANTIES WITH REGARD TO THIS SOFTWARE AND ITS
% DOCUMENTATION, INCLUDING ALL IMPLIED WARRANTIES OF MERCHANTABILITY AND
% FITNESS.  IN NO EVENT SHALL O&A BE LIABLE FOR ANY SPECIAL, INDIRECT OR
% CONSEQUENTIAL DAMAGES, ANY DAMAGES WHATSOEVER RESULTING FROM LOSS OF
% USE, DATA OR PROFITS, WHETHER IN AN ACTION OF CONTRACT, NEGLIGENCE OR
% OTHER TORTIOUS ACTION, ARISING OUT OF OR IN CONNECTION WITH THE USE OR
% PERFORMANCE OF THIS SOFTWARE OR ITS DOCUMENTATION.
%%%%%%%%%%%%%

\chapter{Objects}
\label{ObjectsChapter}


The \library\ was designed with object oriented
concepts.  Each module operates on an abstract entity or object.  Some
modules define the entities while others merely add new operations
for existing objects.  Given that Modula-2 is not an object-oriented
programming language, the methods
by which the objects and their operations are defined varies with 
intended use.  All modules share the same goal of providing access
to new facilities and to operating system features in a portable manner
while retaining the ability to be efficiently implemented.  

This chapter is a guide to the object oriented facilities in the 
library and in Modula-2.  The terms used here may or may not match
the reader's concept of object-oriented programming, therefore 
a definitions section has been included.  The following sections
discuss the problem of object-oriented programming in Modula-2
in general.  Finally, we discuss the module \module{Objects} which
provides support for object-oriented modules.

The realm of object-oriented programming is too large for discussion
here.  If you are unfamiliar with the meaning of object-oriented
design and programming, consult the references.

\section{Terminology}
The following terms are based on the names used in other languages
(particularly SmallTalk), but the definitions have been modified to
better suit Modula-2.  We apologize for the difference in terminology.
After a time, the gentle reader may appreciate the need for these
definitions.

\begin{description}
\item[Object] \index{Object}
    is a type and defines the allowable operations on instances of
    that type.  For example,
    a hammer is an object.  Objects need not be opaque.  The types
    \code{TextIO.Object}, \code{REAL}, and \code{BinaryIO.Index} are objects.
\item[Instance] \index{Instance}
    is a particular incarnation of a type.  An instance is described
    by its state (private memory) and its operations.
    For example, the claw hammer on the table can be used for inserting
    and extracting nails.  The \code{TextIO.Object} just opened is
    read-only and is associated with the file ``fred''.
    The word instance and object are used interchangeablely
    throughout this manual.
\item[Operation] \index{Operation} \index{Method} \index{Message}
    is a procedure which takes an object (sometimes implicitly) and
    performs an action using that object.  An operation may change the
    state of the object's parameter(s) or merely extract information about them
    For example, inserting a nail is an operation which takes a hammer object.
    \proc{TextIO.Read} extracts the next character and advances the
    read pointer from an instance of a {TextIO.Object}.
    Synonyms: method and message.
\item[Class] \index{Class}
    defines an object and its operations.  For example,
    ballpeen is a class of hammer which can insert nails and mold
    metal.  A claw hammer can insert and extract nails, but it can't
    mold metal.  \module{StringBinaryIO}, 
    \module{DosBinaryIO}, and \module{UnixBinaryIO} are all classes.
\item[Category] \index{Category}
    is a group of classes which support a common set of operations.
    For example, hammer is a category.  \module{BinaryIO} is a category.
    Synonyms: super-class and inheritable class.
\item[Visible Objects] \index{Visible Objects}
    are declared publicly and may used as such.
    For purposes of system maintenance, visible objects should be used as
    opaque objects wherever possible.  One of the greatest difficulties
    in porting code occurs where declarations of visible objects are duplicated
    or have been abused.  For example, one should not assume that a
    \code{CHAR} object is 8 bits even though on most implementations it is.
\item[Opaque Objects] \index{Opaque Objects}
    are exported from definition modules and are defined in implementation
    modules.
\item[Private Objects] \index{Private Objects} \index{\code{PRIVATE}}
    are declared visibly for one reason or another,
    but their implementation should be treated as opaque.  In other words,
    private objects are visible objects with a Surgeon General's warning
    against improper usage.
    In the library, instances of private objects must be accessed by
    using the keyword \code{PRIVATE}.
\end{description}


\section{Modula-2}

Given a modicum of programmer discipline,
most HLL provide a good base for OOP.
To aid the programmer in this task, Modula-2 provides:

\begin{description}
\item[Modules] 
    \index{Modules}
    form the basis by which programmers can logically group
    the operations associated with an object(s).  Modularity supports
    maintainability and problem decomposition.

\item[Opaque Types] 
    \index{Opaque Types}
    protect objects and allow for mutable implementations.  
    Protection enhances robustness and maintainability.  The
    ability to change an object's implementation is greatly needed
    in a world of constantly changing specifications.
    
\item[Procedure Types] 
    \index{Procedure Types}
    add customizable extensibility to existing
    implementations.  An importer can drastically change the behaviour
    of an object without affecting the other importers of that object's exporter.

\item[SYSTEM.ADDRESS and SYSTEM.BYTE] 
    \index{SYSTEM.ADDRESS} \index{SYSTEM.BYTE}
    provide extensibility in the
    form of {\em inheritance}.  Generic object modules are used by
    importers to add functionality to objects.  

\end{description}

\subsection{Modules}
\index{Module}

The O\&A Library is divided into modules.  Each module is usually
designed to support a single object, e.g. \module{Strings}, \module{DirIO},
and \module{TextIO}.
A module in the library is a collection of operations for a particular
object and may also define that object.  For example, \module{TextIO}
defines an opaque object and the object's fundamental operations.
The module \module{Strings} defines the fundamental operations for the
generic object \code{ARRAY OF CHAR}.  Although the flavor of the operations
may seem different, these are both good examples of object oriented modules.
In some cases, there is only a single instance of an object per program.
For example, \module{ProgArgs} supports the operations on the
``program arguments'' object for which there is only one instance
per program.  

Many of the modules in the library support more than one object.  For
example, \module{CardIO} supports the \code{CARDINAL} object and the 
\code{TextIO.Object}.  In fact, one of the purposes of the library
is to ``finish'' defining the necessary set of operations for the
generic Modula-2 objects.  Another purpose was to define some new objects
(e.g. error output, textual I/O, and directories) which are ``missing''
in Modula-2.  This is the essence of object-oriented programming.


\subsection{Opaque Types}
\index{Opaque Type}

There are two ways to create new objects in Modula-2: \newterm{opaquely} and
\newterm{visibly}.  Opaque objects (types) are defined in most 
Modula-2 books.  Visible objects are normal type declarations.  The
library defines both kinds of objects.  For example, \code{BinaryIO.Object} is
opaque and \code{BinaryIO.Index} is visible.  Defining opaque objects
enhances system maintenance and thus portability.   For example, an
\code{TextIO.Object} can be implemented in any fashion.  For Unix and 
MS-DOS, they have been implemented using \code{BinaryIO.Object}s, but
on other systems this may not be the case.
Visible objects are usually defined when their implementation is
a necessary semantic to their function, e.g. \code{SysTypes.CARDINAL16}
occupies 32 bits of storage space.  

Private Objects are a special breed, because they are visible but
pretend to be opaque.  This convention was introduced so that importer
could know the size of the object and thus instances of these objects
need not be dynamically allocated.  A good example is the object
\code{SysTypes.CARDINAL32} for the IBM-PC.  In this case, the object
is emulating the generic object \code{CARDINAL} which also does not 
need to be dynamically allocated.  If \code{SysTypes.CARDINAL32} were
not declared as a private object, then each instance would need to
be {\em created} and {\em destroyed}.  Standard Modula-2 assignment 
also would not behave as it would with the generic object \code{CARDINAL}.
Another reason for declaring objects as private instead of opaque is
to make the size known and thus make operations such as I/O simpler.
The object \code{BinaryIO.Index} was designed with this feature in mind.

In all cases, the object structure depends strongly upon
its intended uses and implementation complexity.  Objects declared
in SysTypes are simple to implement even on systems like the IBM-PC,
therefore it makes sense to define them as private or visible.
File I/O is much more complex and thus these
objects are defined as opaque.  Some modules such as \module{DirIO}
do not even define a Modula-2 type to represent a directory, because the
concept of a directory varies so greatly from system to system that
portability would be sharply reduced if such a type were defined.

\subsection{Procedure Types}
\index{Procedure Types}

Modula-2 procedure types are not first-class citizens, but are 
sufficient to support OOP.  Procedure types can implement multi-flavored
objects.  \module{Lists} and \module{CatBinaryIO} use this feature to
allow importer defined objects to be used in conjunction with these
modules. This is a form of \newterm{inheritence} or \newterm{polymorphism}
known in languages like SmallTalk and C++.  Because the Modula-2 form
is not truly polymorphic, we call inheritable object modules
\newterm{categories}.
A category is limited, because
the object (module) must make itself inheritable while in these other
languages the converse is the case.  In Modula-2, creating inheritable
objects requires discipline and examples.

There are few important points to consider when using procedure types:
\begin{itemize}
\item
    Defer assignment of procedure variables until the last possible moment
    in order to enhance flexibility from the importers' perspective.  
\item
    Procedure types should either be declared with an importer supplied
    state as a parameter (called an \code{importerObject} in the library) 
    or have access 
    to per object state.  This feature supports re-entrancy of all modules and
    allows registered procedures to be object-based.  The procedure types
    in \module{DirIO}
    are an example of an \code{importerObject} and the types in
    \module{CatBinaryIO} provide a good example of state information
    provided on a per object basis.
    Note this rule does not apply if there can be only one object per module
    (e.g. \module{ProgErr.InstallTermProc}).
\item \index{registration}
    If a module supports registration of a global procedure
    (e.g. \module{ProgErrOutput}), the previous registrant should
    be supplied as the return value to the registration call. 
    The newly installed procedure may need to call the previous registrant.
    The single {\em install} entry-point ensures the registration order
    is not confused.
\item \index{upcalls} \index{call-backs}
    Calling registered procedures (aka. upcalls and call-backs)
    can cause dead-lock
    (in conjunction with \module{Tasks}) and/or
    corrupted object or global state.
    Before calling a registered procedure (upcall)
    which can possibly call you back, make sure any globally accessible
    state is stable
    and that as few monitors as possible are locked.  Upon return from
    an upcall, the callee should re-evaluate the state and re-enter
    monitors.  Note that the upcalled procedure may do as it wishes
    including destroying the very object being upcalled unless
    a documented requirement forbids specific actions.
\end{itemize}

\subsection{\code{SYSTEM.ADDRESS} and \code{ARRAY OF SYSTEM.BYTE}}

One of the least used features in most Modula-2 programs are the
generic objects \code{SYSTEM.ADDRESS} and 
\code{ARRAY OF SYSTEM.BYTE}.\footnote{
    The library assumes that all systems are byte addressed.  If
    your system is not, then replace all references to \code{BYTE} with 
    \code{WORD} and pray that \code{SYSTEM.TSIZE} and \code{SIZE} return
    amounts in words.
    }
A good example of their use is the \module{FormatIO} family of modules.
A typical example is the procedure \proc{SimpleFIO.Write1} declared
as follows:

\begin{verbatim}
        PROCEDURE Write1(
        format : ARRAY OF CHAR;     (* How to write arg1 *)
        arg1   : ARRAY OF SYSTEM.BYTE   (* Object to be written *)
        );
\end{verbatim}

The \code{arg1} parameter may be any Modula-2 object.
\module{FormatIO} does not know the type of this value, but allows
other modules to register format names and associated procedures
to handle the data.  For example,

\begin{verbatim}
        SimpleFIO.Write1( out, "Index = [Card]", i ); 
        SimpleFIO.Write1( out, "Name: [String]", name );
        SimpleFIO.Write1( out, "Today is [Date]", date );
\end{verbatim}

The language Modula-2 automatically forces the different object types
into the conformant array.  \code{FormatIO} in turn takes the address
and size of the conformant array and passes them to the previously
registered procedure identified in the \code{format}.  The registered
procedure is responsible for verifying the type of the 
argument.\footnote{
    The type verification is weak. Only certain measures (such
    as checking the size) can ensure that the type of \code{arg1} matches
    the type in the \code{format}.  Hence, the genericity is not type-safe.
}

\index{\code{importerObject}}
\code{SYSTEM.ADDRESS} is a generic pointer.  In the library it is
used primarily
to pass importer specified data through modules and back to the original
importer.  For example, \module{NameLists} associates strings with importer
supplied values called \code{importerObjects}.  To add items to a list,
the following procedure is used.

\begin{verbatim}
    PROCEDURE Insert(
        list           : Object;
        key            : ARRAY OF CHAR;
        importerObject : SYSTEM.ADDRESS
        );
\end{verbatim}

The \code{importerObject} is any globally allocated value or 
\code{NameLists.DONTCARE}.
\proc{NameLists.Insert} associates this value with \code{key}.  When
\proc{NameLists.Find} is supplied \code{key}, it will return
the \code{importerObject} with which it was inserted.  The generic
value is untouched by \module{NameLists}.

\section{The module \module{Objects}} \index{memory allocation}

As the previous section demonstrated, most user defined Modula-2 objects
are pointers.  These pointers are for the most part either
dynamically allocated or in module data sections.   Most object-oriented
modules are structured around an opaque object.  The type of this
object is typically called \code{XXX.Object} where \code{XXX} is
some module name, e.g. \code{TextIO.Object}, \code{QStrings.Object},
\code{Lists.Object}.  The objects are generally of a fixed size and
there is usually more than one allocated per program invocation.
The module \module{Objects} was designed to manage the memory
allocation for this particular kind of object.

\subsection{Allocation and Deallocation}
\index{\code{Objects.Manager}}
\module{Objects} supports an object called an \code{Objects.Manager}.
Each instance of an \code{Objects.Manager} (henceforth called a 
{\em manager}) controls the allocation and deallocation for a 
particular class of object.  For example, the implementation
%\module{BinaryIO} creates
a single manager for the allocation and deallocation of
\code{BinaryIO.Objects}.  Here a typical declaration and
use taken from the library:

\begin{verbatim}
    CONST
        moduleName = "BinaryIO";
    TYPE
        Object = POINTER TO ObjectRec;
        ObjectRec = RECORD
            ...
        END;
    VAR
        objectMgr : Objects.Manager;
    BEGIN (* BinaryIO *)
        Objects.CreateManager( objectMgr, SYSTEM.TSIZE( ObjectRec ),
                               moduleName );
\end{verbatim}

The \code{moduleName} is provided for debugging purposes (discussed later).
After creation, the \code{objectMgr} is used in calls to \module{Objects}.

The two procedures \proc{Allocate} and \proc{Deallocate} allow the
importer to allocate and deallocate objects as specified in the
definition module.  In most modules, Allocate and Deallocate will
be called in object creation and destruction procedures, respectively.
The following is a fragment of code from a typical object-oriented module's
creation procedure.

\begin{verbatim}
    BEGIN (* Create *)
        Objects.Allocate( objectMgr, object );
        (* Initialize the object *)
        ...
\end{verbatim}

Note that we assume \code{object} is uninitialized upon entry.  
The size of the object was specified in the call to \code{CreateManager}.
If there is not enough memory available to satisfy this request, 
the caller will be terminated.\footnote{
    The module \module{SafeStorage} is used for all memory allocations.
    \module{SafeStorage} indicates memory shortages.  See this
    module if you are concerned with managing memory at its limit.
    }

The following demonstrates the typical procedure for the deallocation
of an object.

\begin{verbatim}
    BEGIN (* Destroy *)
        IF Objects.IsValid( objectMgr, object ) THEN
            Objects.Deallocate( objectMgr, object );
\end{verbatim}

The procedure \proc{IsValid} (discussed in greater detail later)
is necessary to protect importer's in the event that \code{object}
is uninitialized.\footnote{
    This convention was established to help programmers.  There is
    some doubt as to its usefulness.
    }
If the \code{object} was not allocated with \code{objectMgr}, the
caller is terminated.  Upon return, \code{object} is set to \code{NIL}.

\subsection{Validation and Assertion Checking}

A common programming problem is uninitialized data.  Uninitialized
pointer variables can cause no end of confusion.  The procedures
\proc{Assert} and \proc{IsValid} aid the programmer in the detection
of uninitialized pointers at their source {\em before} memory 
becomes corrupted.  \proc{IsValid} is a safe procedure; it
does not interrupt the caller's flow of control whereas \proc{Assert}
terminates the caller if the specified object is invalid.


One use of \proc{IsValid} can be found in the previous section
in conjunction with \proc{Deallocate}.  Typically, \proc{IsValid}
is used when the importer is providing a higher level validation
service.  Here is an excerpt from the procedure \proc{TextIO.GetError}.

\begin{verbatim}
    BEGIN (* GetError *)
        IF NOT Objects.IsValid( objectMgr, file ) THEN
            RETURN IOErrors.notOpen;
        END;
        RETURN file^.errorState;
\end{verbatim}

\proc{GetError} interrogates the \code{file}'s status.  If the
\code{file} is not open, the caller gets back the error code
\code{IOErrors.notOpen}.  Without the module \module{Objects},
this call would probably end up referencing invalid memory
or terminating the caller if the file was not open.

Assertion checking (aka precondition checking) 
occurs throughout the library in various forms.  The procedure
\proc{Objects.Assert} can be used to assert the precondition
that the object passed is valid.  For example, the following
code fragment is from \proc{TextIO.IsInteractive}.

\begin{verbatim}
    BEGIN (* IsInteractive *)
        @IF Assert THEN
            Objects.Assert( objectMgr, file );
        @END
        RETURN file^.isInteractive;
\end{verbatim}

If \code{file} is invalid (i.e. \proc{IsValid} returns FALSE),
the caller will be terminated with a message of the form:
\begin{verbatim}
    TextIO: invalid object.
\end{verbatim}
where \code{TextIO} is the name which was passed to the \code{CreateManager}
call.

The preprocessor flag \code{Assert}
can be used to control the inclusion or exclusion of the call
to \proc{Objects.Assert}, the trade-off being between faster
code execution and better error detection.  


The difference between \proc{IsValid} and \proc{Assert} is more
than a boolean return value.  \proc{Assert} provides a regular
way of terminating a program as the result of an invalid object.
A feature of \proc{IsValid} is that it {\em never} terminates
the caller accidentally.  This feature turns out to be a handicap
in terms of implementation considerations, as \proc{IsValid}
can take more time to execute than \proc{Assert}.

The following is a list of suggestions for object-oriented
modules:
\begin{itemize}
\item
    All exported procedures should call \proc{IsValid} or \proc{Assert}.
    If this rule is held fast, non-exported procedures do not need
    object assertion checking.
\item
    Procedures which report the status of an object (e.g. 
    \code{GetError}, \code{IsValid}) should call \proc{Objects.IsValid}
    and not \proc{Objects.Assert}.
\item
    All modules which support opaque or private objects should have
    a harmless object initialization procedure (e.g. 
    \code{TextIO.SetInvalid}) and a validation checking procedure
    (e.g. \code{Lists.IsValid}).
\item
    Calls to \code{Objects.Assert} should be controlled by the preprocessor
    flag \code{Assert}.
\end{itemize}

\subsection{Lists of Objects}

\module{Objects} exports the procedures
\proc{NumAllocated}, \proc{MakeFirstNext}, and \proc{Next}.
The first procedure returns the number of objects allocated with
a manager.  The next two procedures allow the
importer to traverse the list of objects associated with a manager.
The list is kept in first allocated, first returned order.  A
sample usage follows.

\begin{verbatim}
    Objects.MakeFirstNext( objectMgr );
    WHILE Objects.Next( objectMgr, object ) DO
        ...Do something with the object...
    END;
\end{verbatim}

Each manager maintains an internal list pointer which can be reset to the
start of the list by \proc{MakeFirstNext}.  Each successive call to 
\proc{Next} returns the next most recently allocated member in the list.
If all the pointers were {\em visible}, one could translate this
loop into:

\begin{verbatim}
    object := objectMgr^.root;
    WHILE object # NIL DO
        ...Do something with the object...
        object := object^.next;
    END;
\end{verbatim}

There are no restrictions on what may be done between calls to 
\proc{Next}; the importer is free to deallocate and allocate
objects with the manager.  \module{Objects} will maintain the
list properly.

\subsection{Implementation and Debugging}

Tracking memory usage in systems without automatic
garbage collection can be a problem.  The procedure \proc{PrintStatus}
displays a list of object statistics by manager.  The
following was generated by calling this procedure.

\begin{verbatim}
        Object Allocation Statistics
Module                            Count   Free   Size
NameLists                             3      0     11
Lists.objects                         3      0     32
Lists.types                           5      0     17
Lists.Anonymous                       5      0     28
Lists.Anonymous                       1      0     28
DosBinaryIO                           3      0    272
Lists.Anonymous                       1      0     28
BinaryIO                              2      1     20
Notices                               2      0     15
Interrupts                            5      0    169
TaskTime                              3      0     17
Monitors.TaskInfo                     3      0     23
TaskMonitors                          3      0     56
TextIO                                2      1     28
TextIO.buffer                         1      1   1034
\end{verbatim}

The \code{Module} column specifies the name of the manager as
passed to \proc{CreateManager}.  \code{Count} indicates the
number of allocated objects, i.e. what the procedure 
\proc{NumAllocated} would return for each of the objects in
the list.  In the last column is the size of the
objects allocated {\em excluding} the overhead required by
\module{Objects}.
The number of \code{free} objects (third column) is interesting.
\module{Objects} keeps a free list of objects per manager for
efficiency.
Summing the free and allocated counts will yield the peak memory
used by a manager, since the last free list purge.  The
notice \proc{Notices.GetOutOfMemoryNotice} is called by 
\module{SafeStorage} to free all unused memory.  This is how the
free lists get purged (returned back to the global memory pool or
heap).
